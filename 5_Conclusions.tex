% Chapter 5
\glsresetall % reset the glossary to expand acronyms again
\chapter{Conclusions and Future Work}\label{ch:Conclusion}

\section{Summary of Conclusions}

Garter is a new teaching programming language, based on the Python Programming
Language. It provides the developer with a safe environment in which they can
learn to write programs, while familiarising themselves with Python syntax and
idioms, making the transition into real programming languages used in production
around the world much easier.

\section{Limitations and Future Work}\label{sec:FutureWork}

Garter unfortunately currently has many limitations, mostly due to the short
timeframe in which it was designed and implemented. We hope to shore up these
problems in a future version of Garter.

\begin{enumerate}
\item Better distribution story

Garter's prototype implementation is not yet easy to distribute to students.
Work must be done to make an implementation as easy to install as a Python
implementation, such that Garter can be used by students without having to
build it from source.

\item Garter lacks a module system.

Garter piggybacks on python, but doesn't currently support module loading,
outside of a small pre-defined set of libraries. A module system to allow
code modularization could be extremely useful for teachers and students,
especially when dealing with larger projects.

\item Garter lacks a mechanism for exposing Python libraries to Garter code

Occasionally a teacher may want to allow their students to perform actions which
are not possible without interacting with other libraries written in Python,
such as interacting with a graphics, windowing, high performance math,
networking or similar library. Garter currently provides no mechanism for
exposing that library other than modifying the distribution directly. Ideally a
mechanism for doing this should be integrated into the language proper.

\item Garter lacks a class inheritance structure

To enter and participate in modern programming, you often need to learn the
mechanics of Object-Oriented Programming, specifically single-inheritance with
interfaces. No mechanism was implemented for doing this in Garter
(See \ref{sec:classdef}), which means that teaching of OOP concepts will have to
occur in a different language. A future Garter may perform the design effort to
add support for inheritance.

\item Learning Resources

Garter currently lacks the learning resources avaliable for languages such as
Python and Turing. These languages have existed for longer, and thus have many
more textbooks etc. written using their syntax and idioms. As Garter is close to
both Python and Turing in language design, textbooks for these languages could
probably be simplified into good Garter teaching material.

\end{enumerate}


