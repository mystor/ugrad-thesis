% Chapter 3
\glsresetall % reset the glossary to expand acronyms again
\chapter{Prototype Implementation}\label{ch:Prototype}

In order to validate the claims of simplicity and feasibility, a prototype implementation of the Garter language was devised.

Many options were considered for how to implement this prototype, mostly focusing on compiling the resulting program back down to Python logic, while performing the validation.
See appendix (Alternative Implementation Strategies)

++ Program Structure

Inserted a phase
cpython (lexer) -> cpython (parser) -> garter (validator) -> cpython (compiler) -> cpython (interpereter) (DIAGRAM)

The strategy which was selected was to modify the cpython interpereter, adding the extra syntax required by the Garter programming language to Python's AST (see language design # assignment for info on extra syntax).
Validation pass was written. Takes python's pre-compilation AST and typechecks it. Written in python for ease of development (over C)

New compiler entry point for python programs, like the built in compile function provided by Python itself. Has same functionality, but also performs typechecking

Modify programs such as IDLE (the Python IDE) to instead use this compile function rather than the built-in one. Means that code run in IDLE is now valid Garter code.

Also created command line REPL which uses this entry point - acts like the Python executable.

+++ Why not replace python completely

Cannot completely replace the standard compile pathways with Garter ones - python code is used internally in python and would need to be re-written.

++ REPL Support

REPLs require maintaining scope and type information across compilations. This is done by exposing
an opaque scope object to calling code of compile function. Can be passed to compile function on
a future pass to make the logic run in the same scope as previous scope.

+++ Modifications to REPL mode

Modifications needed to be made to the repl runners such as IDLE in order to ensure that if runtime error killed statement when it is being executed: changes to types in scope are not recorded.

Otherwise, garter validator might think that x has been initialized, when it actually has not yet.


