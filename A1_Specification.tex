% Chapter 4
\glsresetall % reset the glossary to expand acronyms again
\chapter{Specification}\label{ch:Specification}

\newcommand{\bs}[0]{\textbackslash}

\section{Lexical analysis}

Garter's syntax and lexical analysis are derived from the syntax and lexical rules
from Python. For that reason, many parts of this section are derived directly
from the Python 3 Language Reference \cite{pythonweb}.

Garter programs are described as a series of Unicode code points, and all lexical
analysis is performed on the basis of these Unicode code points.

\subsection{Logical and Physical Lines}

A Garter program is formed of logical lines. The end of a logical line is
represented by the \code{NEWLINE} token. A logical line is formed by one or more
physical line, joined by explicit or implicit joins.

A physical line is a sequence of unicode code points followed by a newline
sequence. This sequence is any of \code{\bs r}, \code{\bs n}, or \code{\bs r\bs n} where \code{\bs r} represents the ASCII Carriage Return (CR) character, and \code{\bs n} represents the ASCII Linefeed (LF) character.

\subsection{Line joining rules}

If the last character in a physical line is the backslash character (\code{\bs}),
and the character is not part of another token (such as a string token), the
current physical line is joined with the next physical line, and no
\code{NEWLINE} token is emitted. Lines which are explicitly joined this way may
not carry comments

If the end of a physical line is reached while inside a pair of parentheses
(\code{()}), square brackets (\code{[]}), or curly braces (\code{\{\}}), the
physical line is implicitly joined with the next line. Lines joined this way may
carry comments.

\subsection{Comments}

A comment starts with the hash character (\code{\#}) which is not part of a
string literal, and ends at the end of a physical line. Comments are ignored by
syntax, and do not cause tokens to be emitted.

\subsection{Indentation}

Leading whitespace at the beginning of a logical line is used to determine the
indentation level of the line, which is used for blocks in the Garter
programming language.

Whitespace measurements are done on spaces, with tabs being replaced by one to
eight spaces, such that the total number of characters is a multiple of 8.

The lexical analysis uses these indentation levels to produce \code{INDENT} and
\code{DEDENT} tokens, using a stack. The following excerpt from the Python
reference explains the algorithm used:

Before the first line of the file is read, a single zero is pushed on the stack;
this will never be popped off again.  The numbers pushed on the stack will
always be strictly increasing from bottom to top.  At the beginning of each
logical line, the line's indentation level is compared to the top of the stack.
If it is equal, nothing happens. If it is larger, it is pushed on the stack, and
one INDENT token is generated.  If it is smaller, it *must* be one of the
numbers occurring on the stack; all numbers on the stack that are larger are
popped off, and for each number popped off a DEDENT token is generated.  At the
end of the file, a DEDENT token is generated for each number remaining on the
stack that is larger than zero.

\subsubsection{Whitespace}

Whitespace between tokens is ignored, unless it contributes to the indentation
rules explained above.

\subsection{Identifiers}
\label{sec:lex_identifiers}

Garter identifiers may be of arbitrary length, and may be formed as follows:

\begin{lstlisting}
identifier ::= start continue*
start      ::= <'a'-'z', 'A'-'Z', '_'>
start      ::= <start, '0'-'9'>
\end{lstlisting}

Implementations may also support additional unicode characters from outside the
ASCII range, following Unicode standard annex UAX-31, like Python.

As an additional restriction, names beginning and ending with two ASCII
underscores (\code{\_\_}), such as \code{\_\_ADD\_\_}, \code{\_\_init\_\_}) are
reserved for implementation use, and may not be written in a Garter program.

\subsubsection{Keywords}

Keywords are a set of identifiers which cannot be used as ordinary identifiers.
Unlike Identifiers, they do not produce a \code{NAME} token, but instead produce
a token specific to the keyword. For any keyword \code{Foo}, the token generated
will be written as \code{'Foo'} in this document.

Some keywords in garter are included for compatibility with the Python
programming language, while others are keywords used only within the Garter
programming language.

The following are the set of keywords in Garter:

\begin{lstlisting}
False      class      finally    is         return
None       continue   for        lambda     try
True       def        from       nonlocal   while
and        del        global     not        with
as         elif       if         or         yield
assert     else       import     pass       print
break      except     in         raise      range
int        float      bool       str        len
\end{lstlisting}

\subsection{Literals}

\subsubsection{String Literals}
\label{sec:string_literals}

String literals takes the following form
\begin{lstlisting}
stringliteral: (shortstring | longstring)
shortstring: ("'" shortstringitem* "'" |
              '"' shortstringitem* '"')
longstring: ("'''" longstringitem* "'''" |
             '"""' longstringitem* '"""')
shortstringitem: shortstringchar | stringescapeseq
longstringitem: longstringchar | stringescapeseq
shortstringchar: <any source character except "\\"
                  or newline or the quote>
longstringchar: <any source character except "\\">
bytesescapeseq: "\\" <any ASCII character>
\end{lstlisting}

Escape sequences in string and bytes literals are interpreted according to rules
similar to those used by Standard C. The recognized escape sequences are:

\begin{center}
  \begin{tabular}{|l|l|}
    \hline
    Escape Sequence & Meaning\\ \hline
    \code{\bs{}newline} & Backslash and newline ignored \\
    \code{\bs\bs} & Backslash (\bs) \\
    \code{\bs{}'} & Single quote (') \\
    \code{\bs{}"} & Double quote (") \\
    \code{\bs{}a} & ASCII Bell (BEL) \\
    \code{\bs{}b} & ASCII Backspace (BS) \\
    \code{\bs{}f} & ASCII Formfeed (FF) \\
    \code{\bs{}n} & ASCII Linefeed (LF) \\
    \code{\bs{}r} & ASCII Carriage Return (CR) \\
    \code{\bs{}t} & ASCII Horizontal Tab (TAB) \\
    \code{\bs{}v} & ASCII Vertical Tab (VT) \\
    \code{\bs{}ooo} & Character with octal value ooo \\
    \code{\bs{}xhh} & Character with hex value hh \\
    \code{\bs{}N\{name\}} & Character named name in the Unicode database\\
    \code{\bs{}uxxxx} & Character with 16-bit hex value xxxx\\
    \code{\bs{}Uxxxxxxxx} & Character with 32-bit hex value xxxxxxxx \\ \hline
  \end{tabular}
\end{center}


All unrecognized escape sequences are left in the string unchanged, including
leaving the backslash in the result.

Multiple consecutive string literals delimited by only whitespace are allowed,
and will be concatenated into a single combined string literal. This allows
writing, for example:

\begin{lstlisting}[language=Python]
x := ("string 1" "string 2")
\end{lstlisting}

Which assigns the value \code{"string 1string 2"} to x.

\subsection{Integer Literals}

The following is the lexical definition for integer literals:

\begin{lstlisting}
integer        ::=  decimalinteger | octinteger |
                    hexinteger | bininteger
decimalinteger ::=  nonzerodigit digit* | "0"+
nonzerodigit   ::=  "1"..."9"
digit          ::=  "0"..."9"
octinteger     ::=  "0" ("o" | "O") octdigit+
hexinteger     ::=  "0" ("x" | "X") hexdigit+
bininteger     ::=  "0" ("b" | "B") bindigit+
octdigit       ::=  "0"..."7"
hexdigit       ::=  digit | "a"..."f" | "A"..."F"
bindigit       ::=  "0" | "1"
\end{lstlisting}

Note that decimal integer literals may not have leading \code{0} characters.
This is to cause c-style octal literals to be an error, as otherwise they would
have unexpected behavior.

\subsection{Floating Point Literals}

Floating point literals expand on integer literals to allow the definition of
non-integral numbers. The following is the lexical definition:

\begin{lstlisting}
floatnumber   ::=  pointfloat | exponentfloat
pointfloat    ::=  [intpart] fraction | intpart "."
exponentfloat ::=  (intpart | pointfloat) exponent
intpart       ::=  digit+
fraction      ::=  "." digit+
exponent      ::=  ("e" | "E") ["+" | "-"] digit+
\end{lstlisting}

\subsection{Operators and Delimiters}

The following are additional special operator and delimiter tokens used by Garter:

\begin{lstlisting}
+       -       *       **      /       //      %
<       >       <=      >=      ==      !=      +=
-=      *=      /=      //=     %=      **=     -> 
,       :       .       ;       =       (       )
[       ]       {       }
\end{lstlisting}

\section{Data Model}

Values in garter are objects. Objects have a type, which defines the type's
fields, and what other operations which may be performed on it. Objects have
identity, and are stored by reference in variables or fields. Assignment doesn't
mutate the object inside of the variable/field, but instead replaces the
reference. A variable or field is immutable if its value cannot be changed.


\subsection{Type Subsumption}
\label{subsumption}

A type \code{T} can be said to subsume another type \code{Q} if a value of type
\code{Q} can be used anywhere that a value of type \code{T} can be used. For
example: an \code{int} is a more specific type than a \code{float}, and it would
be legal to use a \code{int} anywhere that a \code{float} would be required.
Thus we can say \code{float} subsumes \code{int}. In addition, we can say the
inverse, that \code{int} is a subtype of \code{float}.

\subsection{Types}

\begin{lstlisting}
type: 'int' | 'float' | 'str' | 'bool' | NAME |
      '{' type ':' type '}' | '[' type ']' |
      (type | 'None') '(' [typelist] ')'
typelist: type (',' type)* [',']
\end{lstlisting}

\subsubsection{Numbers (\code{int} and \code{float})}

\code{float} is an floating point number. An object of this type may asusme the
value of a double precision floating point value. \code{float} exposes no
attributes. \code{float} is ordered, and can be compared with the comparison
operators \code{<}, \code{>}, \code{<=} and \code{>=}.

\code{int} is an integer. An object of this type may assume the value of an
arbitrary integer value. \code{int} exposes no attributes. \code{int} is a
subtype of \code{float}. \code{int} is ordered, and can be compared with the
comparison operators \code{<}, \code{>}, \code{<=} and \code{>=}.


\subsubsection{Strings (\code{str})}

\code{str} is a unicode string. It may assume the value of an arbitrary length
sequence of unicode codepoints. \code{str} is ordered, and can be compared with
the comparison operators \code{<}, \code{>}, \code{<=} and \code{>=}. \code{str}
exposes the following attributes:

\code{join : str([str])}: The \code{join} attribute is a function object which
concatenates the elements of the argument array, using the implicit self
argument as the seperator. The \code{join} attribute is immutable.

\subsubsection{Booleans (\code{bool})}

\code{bool} is a boolean. It may assume either the value \code{True} or
\code{False}. \code{bool} exposes no attributes.

\subsubsection{Dictionaries (\code{\{K: V\}})}

\code{\{K: V\}} is a one-to-many map from values of type \code{K} to values of
type \code{V}. \code{K} and \code{V} may be arbitrary types. \code{\{K: V\}}
exposes the following attributes. All attributes of \code{\{K: V\}} are
immutable:

\code{pop : V(K, V)}: If the first parameter is a key in the map, Mutates the
map, removing the key from the mapping, and returns the associated value.
Otherwise, returns the second parameter.

\code{setdefault : V(K, V)}: If the first parmaeter is a key in the map, returns
its associated value. Otherwise, mutates the map, inserting tke key-value pair
of the parameters to the map, and returns the second parameter.

\code{get : V(K, V)}: If the first parameter is a key in the map, returns the
associated value. Otherwise, returns the second parameter.

\code{clear : None()}: Mutates the map, removing all key-value pairs.

\code{copy : \{K: V\}()}: Returns a new \code{\{K: V\}}, with a shallow copy
of the key-value pairs from the map.

\code{update : None(\{K: V\})}: Mutates the map, adding all key-value pairs from
the first parameter, overriding any existing conflicting key-value pairs.

\subsubsection{Lists (\code{[T]})}

\code{[T]} is a list of values of type \code{T}. An object of this type may
assume the value of an arbitrary length list of values of type \code{T}.
\code{[T]} exposes the following attributes. All attributes of \code{[T]} are
immutable:

\code{append : None(T)}: Mutates the list, adding the first parameter to the end
of the list.

\code{extend : None([T])}: Mutates the list, adding the elements of the first
parameter to the end of the list.

\code{insert : None(int, T)}: Mutates the list, adding the second parameter
at the index specified by the first parameter, shifting all elements after
the insertion point one element.

\code{remove : None(T)}: Mutates the list, removing the first element for which
\code{it == arg} (where \code{it} is the element, and \code{arg} is the
argument) evaluates to \code{True}.

\code{pop : T(int)}: Mutates the list, removing the element at the index
specified by the first parameter, and returning it. Elements with indexes
greater than the first parameter are shifted down one index.

\code{index : int(T)}: Returns the index of the first element in the list for
which \code{it == arg} (where \code{it} is the element, and \code{arg} is the
argument) evalues to \code{True}.

\code{count : int(T)}: Returns the number of elements in the list for which
\code{it == arg} (where \code{it} is the element, and \code{arg} is the
argument) evaluates to \code{True}.

\code{reverse : None()}: Mutates the list, reversing the order of its elements.

\code{sort : None()}: Mutates the list, sorting it in ascending order. This
attribute is only avaliable on the list types \code{[int]}, \code{[float]}, and
\code{[str]}.

\subsubsection{Functions (\code{T(P, ...)})}

Function objects may have a return type (\code{T}), or if \code{None} is written
instead of a type, the function returns no value. In addition, they have an
arbitrary-length list of parameter types. Function objects do not expose any
attributes.

\subsubsection{User-defined Types}

Users may define their own types using the \code{class} statement (Section
\ref{sec:classdef}). These classes have the attributes as defined by their class
definition. They are unordered, and can be compared for identity with the
equality (\code{==} and \code{!=}) operator.

For more details on user-defined types, see Section \ref{sec:classdef}.

\subsection{Unwritable Types}
\label{writable_types}

A type is said to be an unwritable type if it cannot be written using the type
syntax. Values with unwritable types may not be assigned to variables or used as
the type of fields, parameters, or return values. These types may only be used
as the type of a temporary value within an expression. Unwritable types are the types
of expressions such as \code{None}, \code{[]} and \code{\{\}}.

The expression \code{None} has the unwritable type \code{\_\_NoneType\_\_}. This
type is subsumed by all class types. Thus, the \code{None} value can be used in
place of any class type due to subsumption rules (Section \ref{subsumption}).

The expression \code{[]} has the unwritable type \code{\_\_EmptyListType\_\_}.
This type is subsumed by all list types. Thus, the \code{[]} value can be used
in place of any list type due to subsumption rules (Section \ref{subsumption}).
\code{\_\_EmptyListType\_\_} exposes the same methods as a normal list, except
for those which require knowing the element type.

The expression \code{\{\}} has the unwritable type \code{\_\_EmptyDictType\_\_}.
This type is subsumed by all dict types. Thus, the \code{\{\}} value can be used
in place of any dict type due to subsumption rules (Section \ref{subsumption}).
\code{\_\_EmptyDictType\_\_} exposes the same methods as a normal dict, except
for those which require knowing the key/value types.

All compound list or dictionary types which have keys, values, or item types which
are unwritable are also unwritable. They can be subsumed by any type for which
all literal types match, and the unwritable keys, values, and/or items are subsumed
by the corresponding keys, values, and/or items in the subsuming type.

\section{Program Start Points}

\begin{lstlisting}
single_input: (NEWLINE | simple_stmt |
               compound_stmt NEWLINE |
               defn NEWLINE)
file_input: (NEWLINE | stmt)* ENDMARKER
\end{lstlisting}

A Garter input may either consist of a self-contained program, consisting of a
series of newlines and statements; or a single line of interactive
input coming from, for example, a REPL.

\section{Definitions and Scoping}
\label{sec:defs_and_scoping}

\begin{lstlisting}
defn: funcdef | classdef | vardef
\end{lstlisting}

Definitions may occur wherever a statement is permitted. They bind a value to
the given name in the current scope. Binding is performed on a per-block level.
Shadowing of variables from outside of the current function is permitted, but
shadowing within a function is not allowed. Redeclarations are also prohibited.

When referring to a name, first a check is made to see if the variable is local.
A variable is local if there is a statement declaring that variable within the
current function (whether that variable is currently in scope is ignored). If
the name is nonlocal, enclosing scopes are checked for local variables. If a
variable with the name is not found, it is a validation time error. Once a
variable is found, the variable's validity is checked. A local variable is valid
if it has been declared within the current block or a direct parent. A nonlocal
variable is valid if it is declared above the enclosing function definition in
the root of a function or module. Other references are prohibited as they may no
longer be valid when the function is called.

Variables may not be mutated if they are defined using an immutable declaration
form. Nonlocal variables may not be mutated, although this may be bypassed with
the \code{varfwd} forms described in Section \ref{sec:funcdef}.

\subsection{Function Definition}
\label{sec:funcdef}

\begin{lstlisting}
funcdef: ('def' NAME '(' [paramlist] ')' ['->' type] ':'
          funcbody)
paramlist: param (',' param)* [',']
param: NAME ':' type

funcbody: (simple_stmt | NEWLINE INDENT varfwd* stmt+ DEDENT)
varfwd: globaldef | nonlocaldef
globaldef: 'global' NAME (',' NAME)*
nonlocaldef: 'nonlocal' NAME (',' NAME)*
\end{lstlisting}

A function definition defines a new immutable variable with the given name. It has
the type of the corresponding function object, with the return type specified
after the \code{->} token, or None if no return type is specified. The parameter
types are defined as the types written after the \code{:} token in each parameter.

For example, the function definition:

\begin{lstlisting}
def foo(a: int, b: float, c: str) -> bool:
    ....
\end{lstlisting}

Would define ani immutable variable of type \code{bool(int, float, str)}.

When the function is invoked, the statements in the body of the function are executed
sequentially.

The variable forward forms, \code{globaldef} and \code{nonlocaldef} accept names as
arguments. They state that for name resolution rules, the global, or nonlocal (but in
another function) variables should be treated as though they are local. This means that
they may be mutated, even though they exist in an enclosing nonlocal scope.

Calls to functions are described in more detail in Section \ref{sec:call}.

The body of the function is not validated until either the end of the current
input's validation phase, (either an interactive command or program file), or
the validation of its first reference to the variable within the current input.
This means that functions may refer to other functions or variables which have
not been declared yet, as long as they are declared before the function is used.

\subsection{Class Definition}
\label{sec:classdef}

\begin{lstlisting}
classdef: 'class' NAME ':' class_body
class_body: fielddef | NEWLINE INDENT class_stmt+ DEDENT
class_stmt: fielddef | mthddef

fielddef: NAME ':' [type] '=' expr
mthddef: ('def' NAME '(' mparamlist ')' ['->' type] ':'
          funcbody)
mparamlist: NAME (',' param)* [',']
\end{lstlisting}

A class definition defines a new type to the Garter type system. User-defined
classes may not be referred to in a Garter program before their definition.

When the garter class definition for a type \code{T} is written, an immutable
variable of type \code{T()} is defined with the name \code{T}. This function can
be called to get a new instance of the type \code{T}. In addition, the
initializer statements for the class are run when a class definition is executed
as a statement.

If a class is declared within a function, each time that function is run the
initializer statements for the class will be re-run, and will be used for
new instances of that class within the given scope.

All field and method definitions declare attributes on the class type.

All class types also accept, as a possible value, the value \code{None}. This
represents the absense of a meaningful value. Any attempt to access a field on a
\code{None} value results in a runtime error.

\subsubsection{Field Definitions}
\label{sec:fielddef}
A field definition defines a new mutable attribute with the written type on the
class type. For example, the field definition \code{x : int = 10} would define a
mutable attribute \code{x} on the class type, with type \code{int}. Like with
variable declarations, the type of the field may be inferred if the type is not
written.

\subsubsection{Method Definitions}
\label{sec:methoddef}
A method definition defines a new mutable attribute with a function type on the
class type. The function type will be the same as if the method definition was
written as a function definition (Section \ref{sec:funcdef}), except that the
first parameter need not be typed, and will be implicitly passed a reference to
the instance of the class the function is called on.

Calls are described in more detail in Section \ref{sec:call}.

\subsubsection{Variable Definition}
\label{sec:vardef}

\begin{lstlisting}
vardef: NAME ':' [type] '=' expr
\end{lstlisting}

A variable definition defines a new mutable variable in the current scope. If
the statement is invoked at the global scope, then it defines a new variable in
the global scope. If the statement is invoked within the root of a function, it
defines a new variable inside of that function's scope. The expression is then
evaluated, and assigned to that variable.

If the type is not provided, it is inferred from the type of the expression. This is
occasionally not possible, and a type annotation must be provided (such as with the
\code{None} literal).

\section{Statements}

\begin{lstlisting}
stmt: simple_stmt | compound_stmt | defn
simple_stmt: small_stmt (';' small_stmt)* [';'] NEWLINE
\end{lstlisting}

Control flow in Garter is controlled by statements. These statements can be
categorized into two groups: the single-line small statements, and the
multi-line compound statements.

Definitions are also statements, but are described in Section
\ref{sec:defs_and_scoping}.

\subsection{Small Statements}

\begin{lstlisting}
small_stmt: (expr | assign_stmt | pass_stmt |
             break_stmt | continue_stmt |
             return_stmt | print_stmt)
\end{lstlisting}

\subsubsection{Expr Statement}

The expr statement allows for an expression to be invoked for its side effects,
discarding the resulting value. Usually this is done with a function or method
call. More details on expressions in Section \ref{sec:expr}.

\subsubsection{Assignment Statement}

The basic assignment operator \code{lhs = rhs} assigns the value in \code{rhs}
to \code{lhs}.

Compound assignment operators, \code{+= -= *= /= \%= **= //=} perform their
operation (\code{+ - * / \% ** //} respectively), and the assign the result to
\code{lhs}, evaluating \code{lhs} only once.

The \code{lhs} of the operator must be an assignable target. This is either a
mutable variable, field, array or dictionary index, or array slice. Other
expressions are not legal on the left of an assignment operator. If a field or
variable is immutable (such as is the case with fields corresponding to methods,
and variables corresponding to functions and classes), then it is not a legal
assignable target. If a variable is nonlocal, it is also not a legal assignment
target. To bypass this, the \code{nonlocal} and \code{global} forms may be used
(Section \ref{sec:funcdef}).

\subsubsection{Pass Statement}

\begin{lstlisting}
pass_stmt: 'pass'
\end{lstlisting}

The pass statement does nothing. It exists to enable the definition of empty
blocks.

\subsubsection{Break Statement}
\label{sec:break_stmt}

\begin{lstlisting}
break_stmt: 'break'
\end{lstlisting}

The break statement may only be placed lexically within the for or while
statements. It causes control flow to immediately continue to the next statement
after the for or while statement, not executing the remainder of the current
iteration, and ignoring the normal loop exit conditions.

\subsubsection{Continue Statement}
\label{sec:continue_stmt}

\begin{lstlisting}
continue_stmt: 'continue'
\end{lstlisting}

The continue statement may only be placed lexically within the for or while
statements. It causes control flow to immediately continue to the next iteration
of the loop, not executing the remainder of the current iteration.

\subsubsection{Return Statement}

\begin{lstlisting}
return_stmt: 'return' [expr]
\end{lstlisting}

The return statement may only be placed lexically within a function declaration.
It causes control flow to immediately exit the current function, not executing
the remaining statements in the function. If the return statement is passed an
expression, then it must have the same type as the return type of the function,
and that value is used as the return value of the function. If the return
statement is not passed an expression, then the function must not have a return
type.

\subsubsection{Print Statement}

\begin{lstlisting}
print_stmt: ('print' '(' expr (',' expr)*
             [',' 'end' '=' expr] [','] ')')
\end{lstlisting}

Prints out the result of casting each of the arguments to a \code{str} to the
screen. If this operation would fail, instead prints out a useful debug
representation of the object. By default, each argument's representation is
seperated by an ASCII space character (\code{' '}), and the print statement's
output is terminated with an ASCII newline character (\code{'\bs{}n'}). This
newline character can be replaced by passing the \code{end} 'named argument',
which will instead terminate the string. The \code{end} named argument must be a
\code{str}.

\subsection{Compound Statements}

\begin{lstlisting}
compound_stmt: (if_stmt | while_stmt | for_stmt)
\end{lstlisting}

\subsubsection{If Statement}
\begin{lstlisting}
if_stmt: ('if' expr ':' suite ('elif' expr ':' suite)*
          ['else' ':' suite])
\end{lstlisting}

The if statement is used for conditional execution. It selects exactly one of
the suites to execute by evaluating the expressions one by one until one is
found to evaluate to the boolean value \code{True}. If all expressions evaluate
to \code{False}, then the suite of the else clause, if present, is executed.

All of the expression arguments must be of type \code{bool}.

\subsubsection{While Statement}
\begin{lstlisting}
while_stmt: 'while' expr ':' suite
\end{lstlisting}

The \code{while} statement is used for repeated execution while an expression evaluates
to \code{True}. It will repeatedly test the expression, and if it is
\code{True}, executes the suite. Whenever the expression evaluates to
\code{False}, the loop terminates.

A \code{break} statement (Section \ref{sec:break_stmt}) executed in the suite
terminates the loop immediately. A \code{continue} statement (Section
\ref{sec:continue_stmt}) executed in the suite skips the rest of the suite and
goes back to testing the expression.


\subsubsection{For Statement}
\begin{lstlisting}
for_stmt: ('for' NAME 'in' (expr | range) ':' suite)
range: 'range' '(' expr [',' expr [',' expr]] [','] ')'
\end{lstlisting}

The \code{for} statement is used to iterate over the elements of a list. The
expression is evaluated once, and must have a \code{[T]} type. The suite is then
executed once for each element of the iterator, with the value bound to the
name. This name binding is a local declaration, much like an the assignment
declaration, however it is not valid outside of the body of the for statement
(Section \ref{sec:vardef}.

If the \code{range} form is used instead of the expression, then all of the
expressions must have type \code{int}. In the 1 argument case, the suite is
executed $N$ times, where $N$ is the first argument, with the name bound to each
integer in the range $[0, N)$. In the 2 argument case, the suite is executed
$M-N$ times, where $N$ is the first argument, and $M$ is the second argument,
with the name bound to each integer in the range $[N, M)$. In the 3 argument
case, it acts much like the 2 argument case, except the step is $I$ instead of
1, where $I$ is the 3rd argument.

A \code{break} statement (Section \ref{sec:break_stmt}) executed in the suite
terminates the loop immediately. A \code{continue} statement (Section
\ref{sec:continue_stmt}) executed in the suite skips the rest of the suite and
proceeds to the next item in the list.


\section{Expressions}
\label{sec:expr}

\subsection{Atoms}

\begin{lstlisting}
atom: ('(' expr ')' |
       '[' [exprlist] ']' |
       '{' [dictmaker] '}' |
       'len' '(' expr ')' |
       ('str' | 'int' | 'float') '(' expr ')' |
       NAME | INTEGER | FLOATNUMBER | STRING+ |
       'None' | 'True' | 'False')
exprlist: expr (',' expr)* [',']
dictmaker: expr ':' expr (',' expr ':' expr)* [',']
\end{lstlisting}

Atoms are the building blocks of expressions, and are the base cases for the
recursive definitions of the other expressions.

\subsubsection{Identifiers (names)}
See Section \ref{sec:lex_identifiers} for lexical information on identifiers.
This identifier evaluates to the value of the variable with the given name
in the current scope. The type of this expression is the type of the variable
with the given name.

If there is no variable in the current scope with the given name, it is a
validation-time error.

\subsubsection{Literals}
Literals evaluate to a literal of the given type. The \code{STRING} literal
evaluates to a \code{str} object, The \code{INTEGER} literal evaluates to a
\code{int} object, the \code{FLOAT} literal evaluates to a \code{float}
object, the \code{'True'} literal evaluates to the \code{bool} value
\code{True}, and the \code{'False'} literal evaluates to the \code{bool}
value \code{False}.

The \code{'None'} literal evaluates to the common \code{None} value of
user-defined types. For more information see Section \ref{sec:classdef}.

\subsubsection{Parenthesized Expression}
Parenthesized Expressions evaluate to the value of the contained expression, and
primarially exist as a mechanism for expression precidence and operation orders.

\subsubsection{List literals}

\begin{lstlisting}
list_literal: '[' [exprlist] ']'
exprlist: expr (',' expr)* [',']
\end{lstlisting}

List literals evaluate their contained expressions in order. All expressions
must be of a single type \code{T}. They evaluate to a \code{[T]} value with the
results of the contained expressions as the list elements.

\subsubsection{Dictionary literals}

\begin{lstlisting}
dict_literal: '{' [dictmaker] '}'
dictmaker: expr ':' expr (',' expr ':' expr)* [',']
\end{lstlisting}

Dictionary literals evaluate their contained expressions in order. All
expressions to the left of the colon must be of a single type \code{K}, while
expressions to the right of the colon must be of a single type \code{V}. They
evaluate to a \code{\{K:V\}} value with the results of the expressions to the
left of the ':' as the keys, and the ones to the right as values. Duplicate keys
produce a runtime error.

\subsubsection{Len Expression}

\begin{lstlisting}
len: 'len' '(' expr ')'
\end{lstlisting}

If the expr is a \code{str}, produces the length of the string as an \code{int}.
If the expr is an \code{[T]}, produces the number of items in the list. If the
expr is an \code{\{K: V\}}, produces the number of key-value pairs in the
dictionary. Otherwise, causes a validation time error.

\subsubsection{Typecast Expressions}

\begin{lstlisting}
typecast: ('int' | 'str' | 'float') '(' expr ')'
\end{lstlisting}

The typecast expressions attempt to convert their argument to their type.

All typecast expressions only accept \code{int}, \code{float}, and \code{str}
arguments.

The \code{str} typecast will convert \code{int} and \code{float} objects to
their \code{str} representation. For example, the \code{int} \code{5} will be
converted to the \code{str} \code{"5"}. It will return \code{str} arguments
unchanged.

The \code{int} typecast will truncate \code{float} types to their closest
integer representation, rounded toward \code{0}. It will also attempt to parse
\code{str} types as an \code{int}, causing a runtime error if this fails. It will
return \code{int} arguments unchanged.

The \code{float} typecast will truncate \code{int} types to their closest
floating point representation, which may be less precise than the \code{int}'s
original representation. It will also attempt to parse \code{str} types as an
\code{float}, causing a runtime error if this fails. It will return \code{float}
arguments unchanged.

\subsection{Primaries}
\begin{lstlisting}
primary: fieldref | subscription | call
\end{lstlisting}

\subsubsection{Field Reference}
\label{sec:fieldref}

\begin{lstlisting}
fieldref: primary '.' NAME
\end{lstlisting}

A field reference accesses an attribute of the given object. The type of this
expression is the type of the attribute on the object. If the object lacks
an attribute with the given name, it is a validation-time error.

\subsubsection{Call}
\label{sec:call}
\begin{lstlisting}
call: primary '(' [exprlist] ')'
\end{lstlisting}

A call calls the function object with the passed arguments. The primary must
evaluate to a function object, and the passed arguments must have types which
correspond to the argument types of the function object.

The result type is the return type of the function object. If the function
object has no return type, then this expression has no result.

This is also used to call methods, as methods are defined as immutable fields
with a function object type.

\subsubsection{Subscription}
\begin{lstlisting}
subscription: primary '[' (expr | [expr] ':' [expr]) ']'
\end{lstlisting}

If the first form of subscription is used, the primary must evaluate to a
\code{[T]} or \code{\{K:V\}}. If the primary is a \code{[T]}, then the expr must
be a \code{int}, and the expression will yield the nth element of the list,
where n is the value of the expr. If the primary is a \code{\{K:V\}}, then the
expr must be a \code{K}, and the expression will yield the \code{V} which is
associated with the given key.

If the second form of subscription is used, then the primary must evaluate to a
\code{[T]}, and both expressions, if provided, must be a \code{int}. The first
argument defaults to \code{0}, and the second defaults to the value of
\code{len(primary)}. The expression yields a new list, containing the values in
the list starting at the index of the first argument, and ending before the
element at the index of the second argument.

Negative integer indexes on \code{[T]} are offset from the end of the list, so
\code{-1} is the index of the last element in the list.

\subsection{Comparison Operators}
\begin{lstlisting}
comparison: arith (comp_op arith)*
comp_op: '<'|'>'|'=='|'>='|'<='|'!='|'in'|'not' 'in'
\end{lstlisting}

These operators are not parsed as left-associative, unlike the Arithmetic
Operators below. Instead, a sequence of comparison operators such as \code{a < b
  == c > d} will be resolved like \code{(a < b) and (b == c) and (c > d)},
except that each expression will only be evaluated once.

These operators all produce an expression of type \code{bool}, and will be
written with only two operands. Their semantic meaning in sequence is as is
written above. Their valid types will be written in place of their operands. As
\code{int} is subsumed by \code{float}, it may be used whenever a \code{float}
is required.

\code{float < float}: \code{True} if $lhs < rhs$, \code{False} otherwise.

\code{float > float}: \code{True} if $lhs > rhs$, \code{False} otherwise.

\code{float <= float}: \code{True} if $lhs \le rhs$, \code{False} otherwise.

\code{float >= float}: \code{True} if $lhs \ge rhs$, \code{False} otherwise.

\code{float == float}: \code{True} if $lhs = rhs$, \code{False} otherwise.

\code{str == str}: \code{True} if $lhs$ and $rhs$ represent the same unicode
sequence, \code{False} otherwise.

\code{bool == bool}: \code{True} if $lhs$ and $rhs$ have the same value,
\code{False} otherwise.

\code{[T] == [T]}: \code{True} if $lhs$ and $rhs$ have the same number of
elements, and for all elements $i$, \code{lhs[i] == rhs[i]}.

\code{\{K: V\} == \{K: V\}}: \code{True} if $lhs$ and $rhs$ have the same set of
keys, and for each key $k$, \code{lhs[k] == rhs[k]}.

\code{T == T}: \code{True} if $lhs$ and $rhs$ have the same identity.

\code{T != T}: \code{True} if \code{T == T} is \code{False}, \code{False}
otherwise.

\code{T in [T]}: \code{True} if there is an element $i$ in $rhs$ such that
\code{lhs == rhs[i]}. \code{False} otherwise.

\code{K in \{K: V\}}: \code{True} if the key $lhs$ is in $rhs$, \code{False}
otherwise.

\code{T not in [T]}: \code{False} if there is an element $i$ in $rhs$ such that
\code{lhs == rhs[i]}. \code{True} otherwise.

\code{K not in \{K: V\}}: \code{False} if the key $lhs$ is in $rhs$, \code{True}
otherwise.

\subsection{Arithmetic Operators}
\begin{lstlisting}
arith: mult (('+'|'-') mult)*
mult: unary (('*'|'/'|'%'|'//') unary)*
unary: ('+'|'-') unary | power
power: primary ['**' unary]
\end{lstlisting}

These operators will be written with their valid types in place of their
operands. Any unlisted type-operand combinations are validation-time errors.
With any operators which accept \code{float} also accept \code{int} in place of
their operands, unless a more specialized operator is avaliable. For example
the \code{-} binary operator accepts any combination of \code{int} and
\code{float} operators, and produces an \code{int} if both operands are
\code{int}, and a \code{float} otherwise, but only has two entries on this list,
one for \code{int - int}, and one for \code{float - float}.

\code{int + int}: Returns the value $lhs + rhs$ as an \code{int}.

\code{float + float}: Returns the value $lhs + rhs$ as a \code{float}.

\code{str + str}: Returns a \code{str} containing the concatenation of the two strings.

\code{[T] + [T]}: Returns a new \code{[T]} containing the concatenation of the two lists.

\code{int - int}: Returns the value $lhs - rhs$ as an \code{int}.

\code{float - float}: Returns the value $lhs - rhs$ as a \code{float}.

\code{int * int}: Returns the value $lhs \times rhs$ as an \code{int}.

\code{float * float}: Returns the value $lhs \times rhs$ as a \code{float}.

\code{float / float}: Returns the value $\frac{lhs}{rhs}$ as a \code{float}.

\code{int \% int}: Returns the remainder from the division $\frac{lhs}{rhs}$ as
an \code{int}. The sign of this remainder will match the sign of the second
operand.

\code{float \% float}: Returns the remainder from the division $\frac{lhs}{rhs}$
as an \code{float}. The sign of this remainder will match the sign of the second
operand.

\code{int // int}: Returns the value $\frac{lhs}{rhs}$, rounded down to the
nearest integer as an \code{int}.

\code{float // float}: Returns the value $\frac{lhs}{rhs}$, rounded down to the
nearest integer as an \code{float}.

\code{float ** float}: Returns the value $lhs^{rhs}$, as a \code{float}.

\code{+ int}: Returns its operand unchanged.

\code{+ float}: Returns its operand unchanged.

\code{- int}: Returns the negation of its operand.

\code{- float}: Returns the negation of its operand.

\subsubsection{Unary Operators}

\code{+}: If the operand is \code{int} or \code{float}, returns it unmodified.
Other operand types cause a validation time error.

\code{-}: If the operand is \code{int} or \code{float}, returns its negation with
the same type. Other operand types cause a validation time error.

\subsection{Boolean Operators}

\begin{lstlisting}
or_expr: and_expr ('or' and_expr)*
and_expr: not_expr ('and' not_expr)*
not_expr: 'not' not_expr | comparison
\end{lstlisting}

Boolean Operators, unlike the other operators, do not necessarially evaluate
all of their operands, they may short-circuit their execution if the value of
their left operand is sufficient to compute the result.

All Boolean operators require all operands to have type \code{bool}.

The \code{or} operator evaluates its left operand. If it produces the boolean
value \code{True}, the result is \code{True}. Otherwise, it evaluates its
right operand, and produces its result.

The \code{and} operator evaluates its left operand. If it produces the boolean
value \code{False}, the result is \code{False}. Otherwise, it evaluates its
right operand, and produces its result.

The \code{not} operator evaluates its operand. If it produces the boolean value
\code{False}, the result is \code{True}. Otherwise, the result is \code{False}.

\subsection{If Expression}
\begin{lstlisting}
expr: or_expr ['if' or_expr 'else' expr]
\end{lstlisting}

The first and third operands must have a common type \code{T}. The second operand
must have type \code{bool}.

The \code{if} expression evaluates its second operand. If it produces the
boolean value \code{True}, the first operand is evaluated, and its result is
produced. Otherwise, the third operand is evaluated, and its result is produced.

\section{Builtin Functions}
\label{sec:functions}

Garter defines only one function in the global scope at the start of the program:

\code{input : str()}: Reads in a line of textual input from the user, producing
a \code{str} containing the read-in text.

\code{ord : int(str)}: Returns the ordinal number for the passed-in character.

