% Chapter 3
\glsresetall % reset the glossary to expand acronyms again
\chapter{Design}\label{ch:Design}

For this thesis, we wanted to explore how we could create a better programming
language for teaching new programmers. Of the requirements we identified in
Section \ref{sec:requirements}, Existing programming languages for teaching such
as Turing \cite{turingpaper} for the most part fulfil the requirements of being
simple to start in, having few confusing edge cases, and restricting the types
of programs which can be written to those which are correct and don't fail at
runtime. Unfortunately, many of them fail when it comes to being a good stepping
stone for students into producting programming languages which people in the
industry use to solve real problems. It is important to reduce the barriers
between the language which people learn in school and solving real-world
problems, as one of the best ways to learn programming is to get engaged in
trying to solve a problem you are facing, which is much easier when you have
access to the libraries and resources provided to and by production programmers.

To satisfy this requirement, Garter was designed as a safe subset of the Python
programming language \cite{pythonweb}. Python is a popular programming language
for modern development, ranking as the 5th most popular language on the GitHub
code sharing platform in August 2015 \cite{githublangs}. The goal was to make
the two languages so similar that it would be trivial for an eager student to
transition their knowledge from writing Garter programs into solving real
problems in Python using the substantial suite of libraries and tools which
Python provides.

In this chapter the design decisions which were made in establishing the Garter
subset of Python are described.

\section{Making Python Type Safe}

One of the desirable properties of a teaching programming language is that it
prevents new developers from writing incorrect programs, and guides them away
from making simple mistakes, especially when these problems only occur at
runtime, and can thus be vary hard to diagnose.

\begin{lstlisting}[language=Python]
x := 5
s := 'x = ' + x
\end{lstlisting}

The above program is written in Garter, and would produce a validation error,
reporting an \code{OperatorTypeMismatch}, \code{The operands to operator '+'
(str + int), are invalid.} The equivalent program in Python would not fail until
runtime, when a \code{TypeError} would be raised. If this code was in an
infrequently traversed code path, such as a failure path, it could remain
undetected, meaning that the program is subtly wrong. In the development of the
Garter prototype, I often accidentally performed this very error within the
error handling code, and didn't catch them until I wrote the test cases for
validation errors.

In addition, we want to prevent code which, while not technically incorrect,
is likely to cause problems in the future. For example, Python allows for
hetrogenous arrays, consisting of values of varying types. This means it's
possible to write code such as:

\begin{lstlisting}[language=Python]
arr := [1, 2, 3, 4, 5, 6, 7, 8, 9]
arr.push(input('Enter a number: '))
\end{lstlisting}

While this code is technically correct in Python, and would execute correctly as
a Python program, future code which expects the elements in \code{arr} to
consist only numbers would fail, as one of the elements is a string. In Garter,
this would produce a \code{ParameterTypeMismatch} validation error, as arrays
are required to be homogenous, which is usually what is desired, such that
individual elements can be treated uniformly.

This means that if you write code like the following in Garter, and it
validates, you can always depend on it working, and there not being any strings
which cause runtime failures far from the source of the problem.

\begin{lstlisting}[language=Python]
sum := 0.0
for elt in arr:
    sum += elt
\end{lstlisting}

To fix this problem, we adapted a simple type system based languages such as
Pascal and Turing. The type system modified to use Python's basic types, such
as \code{int}, \code{float}, \code{bool}, \code{str}, lists (\code{[T]}), dicts
(\code{\{K:V\}}), functions (\code{R(A, R, G...)}), and user defined classes.
This type system limits what values are allowed to exist in the language, such
that only the common use cases, such as homogenous arrays and dictionaries, are
permitted. This means that new developers will be guided away from writing code
which accidentally creates one of these non-homogenous data types or variables,
which is usually an error, even in real Python code.

In Garter we also decided against implementing generic or templated types. The
dict and list types are instead given special status. This was because we
decided that the complexities and design tradeoffs which are involved in this
complex feature were unnecessary, especially for early learning. Instead, the
use of the built-in types is heavily encouraged. Some programming languages
which are used in the industry, such as the Go Programming Language also lack a
generics system, which helped us make this decision \cite{golangweb}.

The possibility of merging the \code{int} and \code{float} types into a single
\code{num} type was considered, however this would have confusing properties
when it comes to array indexing, and what values are OK for using there. In
Python, only \code{int} values can be used for indexing into an array. This
behavior of treating the value \code{2} differently than \code{2.0} would be
very confusing to a new programmer, so the decision was made to formalize the
difference to avoid confusion and common programming errors, especially related
to division of integers.

\begin{lstlisting}[language=Python]
arr[2] # OK
arr[2.0] # Not OK
arr[4/2] # Not OK
arr[4//2] # OK
\end{lstlisting}

\section{Objects}
Data in Garter are represented by Objects. 

\subsection{Value passing style}
Object types can be split into two categories. The primitive
types: \code{int}, \code{float}, \code{str}, \code{bool} and functions have
immutable values, which means that they act as though they are passed by value.

\begin{lstlisting}[language=Python]
x := 5
y := x
x = 10
print(y) # 5
\end{lstlisting}

In contrast, the other types, such as \code{[T]}, \code{\{K: V\}} and user defined types
are not immutable, and act as though they are passed as a reference to a shared object.

\begin{lstlisting}[language=Python]
arr := [1, 2, 3]
arr2 := arr
arr.push(4)
print(arr2) # [1, 2, 3, 4]
\end{lstlisting}

If a new copy of a list or dictionary is required, it can be copied with
the \code{.copy()} method.

\begin{lstlisting}[language=Python]
arr := [1, 2, 3]
arr2 := arr.copy()
arr.push(4)
print(arr2) # [1, 2, 3]
\end{lstlisting}

This behavior is inherited from Python, and directly mimics the behavior used in
many programming languages, where simple types are immutable but complex ones
are mutable and shared by reference.

\subsection{Functions and Methods}

Functions in Garter are implemented as objects, which can be passed around and
have the type \code{R(A, R, G...)}, where \code{R} is the return type, and
\code{A}, \code{R}, and \code{G} are the argument types. The following is the
representation of the types of some functions

\begin{lstlisting}[language=Python]
def foo(): ...                         # None()
def bar(x : int) -> int: ...           # int(int)
def baz() -> int: ...                  # int()
def quxx(x : int, y : int) -> int: ... # int(int, int)
\end{lstlisting}

This decision was made in order to enable teachers to teach some basic
functional programming styles, which are becoming more popular in modern
programming due to the popularity of web programming with languages such as
JavaScript which often use callbacks and other functions as values.

Methods are represented as immutable function attributes on objects, except with
the an implicit \code{self} argument.

\begin{lstlisting}[language=Python]
class C:
    s := 'hi'
    def f(self): print(self.s)
x := C()
y : None() = x.f # Valid
y() # prints hi
\end{lstlisting}

This conforms with Python's representation of Methods, and avoids confusion
around methods, especially when using them in more functional programming styles
which some teachers may choose to use to teach their students.

\section{Expressions}

Expressions are based on the expression forms from Python. Like in Python, all
expressions are evaluated eagerly, which both conforms with the current popular
design for production languages, and simplifies reasoning about when code is
executed. 

\subsection{Literals}

Garter supports the following literals:
\begin{itemize}
\item list literals (e.g. \code{[1, 2, 3]})
\item dict literals (e.g. \code{\{'a': 1, 'b': 2\}})
\item numeric literals (e.g. \code{5}, and \code{5.5})
\item bool literals (\code{True}, and \code{False})
\item str literals (e.g. \code{'foo'}, and \code{"bar"})
\end{itemize}

The list and dictionary literals especially enable much shorter code than
requiring them to be built from the empty objects using mutation methods. This
is especially the case for lookup tables etc.

\subsection{Function Calls}

Function calls are performed with the standard \code{f(x)} syntax. All functions in
Garter take a fixed number of arguments of fixed types, and produce a result of
a fixed type. Python's default arguments and named arguments were avoided in order to
simplify writing and calling functions for new programmers.

\begin{lstlisting}[language=Python]
def f(x : int): print('hi', x)
f(5) # prints 'hi 5'
\end{lstlisting}

\subsubsection{Magic Functions}

Garter also adapted some functions from Python which require variable count,
variable type, or keyword arguments. These functions were special cased, and
built-in to the language, and are present because of how important they are.

The functions which act this way are:

\begin{itemize}
\item \code{len()} - length of the argument (\code{str}, \code{[T]}, or \code{\{K: V\}})
\item conversion operators (\code{int()}, \code{str()}, \code{float()}) - convert the value to the written type
\item \code{input()} - read in a line from the console, as a string. Optionally takes a prompt strin
\end{itemize}

The \code{print()} function-like statement (Section \ref{sec:print_stmt}),
and \code{range()} function-like form (Section \ref{sec:forstmt}) are also
examples of special-cased functions from Python in Garter.

\subsection{Arithmetic}

Garter supports the standard binary operators (\code{+ - * / // \% **}, mapping
to the operations of addition, subtraction, multiplication, division, flooring
division, modulus, and exponentiation) on both \code{int} and \code{float}.

In addition, the \code{+} binary operator is supported on \code{str}
and \code{[T]}, representing string or list concatenation. It produces a new
value with the value of the concatenation of its arguments.

\code{int} and \code{float} also support the unary \code{+} and \code{-} operators,
which do nothing and negate the value respectively.

This set of operators is fairly standard in programming languages, and is copied
from Python. The decision to seperate flooring division from standard division
(which always produces a \code{float}) is based on Python 3's behavior, and also
helps to prevent errors for new developers which don't understand integer
division.

\subsection{Comparison Operations}

Equality with the equality operators (\code{==}, and \code{!=}) are performed by
value, and supported by all types in Garter. As it is done on value, two lists
or dicts with the same elements, but different identities are still equal.

\begin{lstlisting}[language=Python]
arr1 := [1, 2, 3]
arr2 := [1, 2, 3]
print(arr1 == arr2) # True

dict1 := {'a': 1, 'b': 2}
dict2 := {'a': 1, 'b': 2}
print(dict1 == dict2) # True
\end{lstlisting}

This decision was made as it makes the most intuitive sense. However, there are
exceptions to the comparison-is-by-value rule. Namely, user-defined classes and
functions are compared by reference. 

\begin{lstlisting}[language=Python]
class Foo:
    x := 5
a := Foo()
b := Foo()
print(a == b) # False

def a(): print('hi')
def b(): print('hi')
print(a == b) # False
\end{lstlisting}

The other comparison operators, \code{<}, \code{>}, \code{<=}, and \code{>=}
are supported on the \code{str}, \code{int}, and \code{float} types.

Unlike some other programming languages, comparison operations in Garter are not
binary, but rather n-ary. This means that the operation \code{a < b < c} is not
evaluated as either \code{(a < b) < c} or \code{a < (b < c)}. Instead, it is
evaluated as \code{(a < b) and (b < c)}. This is much closer to the mathematical
meaning of that statement as a constraint, and can help prevent issues when
people with mathematical backgrounds try to write constraints which occur in
other languages. This is important as it helps new programmers which may already
be familiar with mathematical notation to adapt more quickly.

Multiple operators can be mixed together in these comparisons, meaning that
the expression \code{a < b == c < d} is legal in Garter.

\subsection{Attributes and Subscription}

Objects often have attributes, which can be accessed using the \code{.} operator.
This is used to access properties such as methods on built-in and user-defined
types, as well as accessing and potentially modifying attributes on user-defined
types.

\begin{lstlisting}[language=Python]
class Foo:
    x := 5
a := Foo()
a.x = 10
print('a.x =', a.x)
\end{lstlisting}

When accessing values in arrays and dictionaries, subscripting syntax is used.
This involves placing the indexing value in square brackets after the value.
\code{int} is the type used for indexing into \code{[T]}, while \code{K} is the
type used for indexing into \code{\{K: V\}}. Lists in Garter are 0-indexed,
which is the most common indexing strategy in use today, as well as the strategy
used in Python.

\begin{lstlisting}[language=Python]
x := {'a': 5}
y := [5, 10]

x['a'] # 5
y[0] # 5
\end{lstlisting}

Lists may also be sliced into over a range, using the slicing syntax
(\code{[S:E]}). Both the values \code{S} and \code{E} are indexes, and can be
omitted, defaulting to the start and end of the array respectively.

Slices can be assigned to with another array. This will update the values in the
slice to be equivalent to the values from the new array, potentially shifting
around values in the original array.

For more details on this syntax, see \ref{sec:spec_subscription}.

\section{Statements}

\subsection{Functions}
Functions in Garter are defined with the \code{def} keyword. Their syntax is taken
directly from Python, including the type annotation syntax. The Type annotation
syntax is based off of PEP 3107 -- Function Annotations \cite{pythonfuncannot}.
We re-used this syntax as it was already part of the Python language, and thus
was guaranteed not to conflict with other important features, and avoids
adding new syntax as much as possible, even though it is extremely infrequently
used in Python projects.

\begin{lstlisting}[language=Python]
def foo(x : int, y : str) -> bool:
    ...
\end{lstlisting}

Function bodies have a seperate scope, and can be nested. They can reference
variables from their enclosing scope, however, they cannot assign to those
variables without declaring the intention with the \code{global} or
\code{nonlocal} declarations (See \ref{sec:funcdef}).

This helps new developers avoid errors, by requiring them to clarify their
intentions as to whether they want to create a local variable, or mutate an
outer variable, before they use assignment statements.

The \code{return} statement exits a function early, and can be passed a value
when the function is defined to produce a result. The \code{return} statement is
required for functions which provide a result on all code paths.

The variable bindings created by function declarations are immutable, and cannot
be re-assigned to.

\subsection{Control Flow Statements}

\code{for} loops in Garter loop over lists or ranges of numbers. While this is more
restrictive than a traditional range statement, it tends to be more in line with
the standard use of the \code{for} loop. (See \ref{sec:forstmt})

\begin{lstlisting}[language=Python]
for x in range(10):
    # x is 0, 1, 2, 3, 4, 5, 6, 7, 8, 9. Declares x only in scope
    ...

for x in [1, 2, 3]:
    # x is 1, 2, 3. Declares x only in scope
    ...
\end{lstlisting}

Garter also supports the \code{while} loop, which is a basic loop. Every
iteration, including the first iteration, its condition is tested. If it
is \code{False}, then the loop aborts, otherwise it continues
(See \ref{sec:whilestmt}).


The \code{break} and \code{continue} statements can be placed within both
the \code{for} and \code{while} looping construct bodies. \code{break} aborts
the innermost loop, continuing at the next statement after the
loop. \code{continue} causes the next item in the \code{for} loop to be iterated
through, and re-triggers the check in a \code{while} loop.

Garter also supports the \code{if} statement. This statement, like in other
languages, tests a boolean condition. If it is \code{True}, one branch is taken,
otherwise the other branch is taken.

\subsection{Assignments}

In Garter, unlike Python, you cannot assign to a variable which has not been
declared. Assignments are performed with the \code{x = 5} statement. 

Variable declarations are performed with the \code{x := 5} or \code{x : int = 5}
statement. these declarations are lexically scoped, but are not permitted
to shadow other bindings with the same name from the same function.

\begin{lstlisting}[language=Python]
def foo():
    if cond:
        x := 5
    else:
        x := 10
    return x # not OK
    # x is defined within the if's branches, not at the root
\end{lstlisting}

\begin{lstlisting}[language=Python]
def foo():
    x := 0
    if cond:
        x = 5
    else:
        x = 10
    return x # OK
    # x defined in same scope as return, only assigned to in if statements
\end{lstlisting}

\begin{lstlisting}[language=Python]
def foo():
    x := 0
    if cond:
        x := 5 # Not OK
        # This is an error, and catches a problem
        # caused by probably unintentional shadowing
\end{lstlisting}

\begin{lstlisting}[language=Python]
x := 0
def foo():
    x := 10 # OK
    # x is in a different function. Doesn't change global x
\end{lstlisting}

\begin{lstlisting}[language=Python]
x := 0
def foo():
    global x # global statement is required to change x
    x = 10
\end{lstlisting}

Garter also supports closures, allowing functions to access variables defined
in the root of an enclosing function or in the root of the program.

\begin{lstlisting}[language=Python]
def foo() -> int():
    x := 5
    def bar() -> int:
        return x
    return bar
foo()() # 5
\end{lstlisting}

Nested functions are nice for teaching basic functional style programming which
is becoming more popular due to the prevaliance of callback programming with
closures in languages like JavaScript.

Accessing values which aren't in the root of a function, such as the loop
iterator value from a \code{for} loop, is not allowed. This is because the value
may not be valid by the time the function is actually called.

\begin{lstlisting}[language=Python]
def foo(a : [int]) -> [int()]:
    out : [int()] = []
    for x in a:
        def bar() -> int:
            return x # Not OK
            # x is not bound in root of function and thus cannot
            # be seen from enclused function the value x may not
            # have the same value as you are expecting by the time
            # the function is done. The same variable is used
            # for all iterations of the loop, so if this was allowed
            # all functions would return the last value itertated
            # through
        out.push(bar)
    return out
\end{lstlisting}

The declaration syntax (\code{x := a}) is the only syntax addition to Python.
It adds no new semantic meaning to Python, having the same meaning in Python
as a bare assignment statement. The meaning is only useful for the Garter
validation pass.

\subsection{Classes}

Users in Garter are allowed to define their own types, with the \code{class}
construct.

\begin{lstlisting}[language=Python]
class C:
    field := initialvalue

    def method(self):
        # self is implicitly avaliable here as the instance of C!
\end{lstlisting}

When a class is defined, it also declares a function with the name of the class.
This function, when called, will create a new instance of that class.

\begin{lstlisting}[language=Python]
x := C()
\end{lstlisting}

One of the unfortunate design decisions made in Garter because of its Python
legacy is the semantics of field default values. The initial values for fields
are evaluated just once when the class is defined. This can lead to confusing
results when combined with list or dictionary fields, and mutation.

\begin{lstlisting}[language=Python]
class C:
    arr := []
x := C()
x.arr.push(5) # Changes value for default value for arr
y := C()
y.arr # Contains 5
\end{lstlisting}

In Python, the normal way to implement this type of logic would be to define the
magic \code{\_\_init\_\_} method on the class, which is called when a new
instance is created. This would mean that the \code{[]} literal would be
evaluated each time that an instance is created.

\begin{lstlisting}[language=Python]
class C:
    def __init__(self):
        self.arr = []
\end{lstlisting}

This syntax is highly unfortunate, requiring an understanding of constructors
etc. Instead, the syntax of creating type-associated values was chosen, as it
has reasonable semantics, although it sometimes creates undesirable results
when used with mutable values.

There was a consideration early in the development of considering consecutive
\code{:=} statements in classes as being an implicit \code{\_\_init\_\_}
statement. This was chosen to be too different semantically from Python, and
was not chosen as the option.

\subsection{Expressions}

Expressions can also just be written as a statement, and they will be evaluated
for their side effects.

\section{Programs}

A program is just a series of statements. They are executed from top to bottom.
Garter also supports interactive evaluation, in which partial programs are
provided to the validator. Garter programs are validated in a single pass from
top to bottom. Most statements are validate in order. Functions don't have their
bodies typechecked until either:

\begin{enumerate}
\item their binding's name is referenced (for example, to be called), or
\item The end of the current program's input is reached.
\end{enumerate}

This allows mutual recursion, for example:

\begin{lstlisting}[language=Python]
def foo(b: bool):
    if b:
        print('b is true')
    else:
        print('b is false')
        bar() # foo is defined before bar is defined

# if foo was called here, it would be a lookup error, but it isn't called until after so it's OK

def bar():
    foo(True)

foo(False)
\end{lstlisting}

\section{Modules}

Garter only supports a small set of basic modules provided by the base language.
The currently supported modules are \code{random} and \code{turtle}. Expansion
to include support for user- and teacher- defined modules are part of future
work \ref{sec:FutureWork}.

Another important part of teaching would be to provide a simple graphics API
for teaching students, such that they can create simple games, and visualize
progress. Designing such an API is, however, outside the scope of this thesis.

