% Chapter 3
\glsresetall % reset the glossary to expand acronyms again
\chapter{Design}\label{ch:Design}

Made decision to use Python because it is popular in industry.


++ Making Python Type Safe

We want to prevent writing incorrect programs

x := 5
s := 'x = ' + x

And guide people away from mistakes

arr := [1, 2, 3, 4, 5, 6, 7, 8, 9]
arr.push(input('Enter a number: '))

Simple Type system from languages like pascal.  Fitted onto Python.
Limits what is allowed (e.g. homogenous arrays). Good thing

Types like arrays and dicts are special, no concept of generic types
Too complex and likely to make confusing errors for beginning. Unnecessary syntax which would have to be explained.


++ Objects

Values are objects. int, float, str, bool and functions are immutable, and are passed by value.
Other types are passed as a pointer to shared memory.

arr := [1, 2, 3]
arr2 := arr
arr.push(4)
print(arr2) # [1, 2, 3, 4]

int and float numeric types are kept seperate to prevent problems with array indexing

arr[2] # OK
arr[2.0] # Not OK
arr[4/2] # Not OK
arr[4//2] # OK

+++ Methods

Methods are just functions as attributes on objects:

class C:
    def f(self): print('hi')
x := C()
y : None() = x.f # Valid
y() # prints hi

++ Expressions

Expressions are taken from Python. Eagerly evaluated for simple reasoning.
Type annotations unnecessary in expressions.

+++ Literals

supports array [1, 2, 3] obejct {'a': b} numeric 5, 5.5 bool True, False and string 'foo' literals
Requires less creation of data for lookups etc. manually with for example init functions.

+++ Function Calls

Performed with standard f(x) syntax
functions take fixed number of arguments, with fixed types, and produce fixed type result
def f(x : int): print('hi', x)
f(5) # prints 'hi 5'

Functions are values, and can be passed around
def f(): print('hi')
x := f
x() # prints 'hi'

++++ Magic Functions

Some 'function'-like expressions are magic and built in. They don't follow
the rules of taking a fixed number of arguments with fixed types:

len(), conversion operators (int(), str(), float()), input()

(SEE ALSO print() magic function-like statement)

+++ Arithmetic

Supports + - * / // \% ** (plus, minus, mult, divide, floor divide, modulus, and exponentiation)
on numbers (int/float)

+ is also supported on arrays and strings - represents array/string concatenation.
+ on arrays produces an new array with a new identity, mutations on new array don't affect old and vice versa.

+ and - are also unary operands on numbers

^ Fairly standard mathematical operators. Easy to learn

+++ Comparison Operations

== is done on value, 'asd' == 'asd' is always true
['a', 's', 'd'] == ['a', 's', 'd'] is also true, even though have diff identities and mutable
^ More intuitive idea of equality: has same value

Exceptions: classes, and functiosn

class Foo:
    x := 5
a := Foo()
b := Foo()
a == b # False

def a(): print('hi')
def b(): print('hi')
a == b # False

+++ Attributes + Subscription

Can access attributes of classes defined. Done with . operator

class Foo:
    x := 5
a := Foo()
a.x = 10
print('a.x =', a.x)

Also can subscript into arrays/dicts

x := {'a': 5}
y := [5, 10]

x['a'] # 5
y[0] # 5 <- 0 indexing is standard in programming languages today

++ Statements

+++ Functions

defined with def. Can be annotated with types.

Type annotation syntax defined in Python in RFC #ref. Used by other project mypy
Re-used in Garter, as it fits well into Python.

def foo(x : int, y : str) -> bool:
    ...

functions are defined as objects
have their own scope where variables are defined. can refer to enclosing variable scope
but cannot assign to it without `global` or `nonlocal` (for nested functions) statement.

Can leave function early with return statement. Return statement is required for function
with result to provide the result, on all code paths.

Function declarations are immutable.

+++ Control Flow

for loops go over lists, or ranges, rather than having manual increments
Simpler, frequently what you were trying to do

for x in range(10): # x is 0, 1, 2, 3, 4, 5, 6, 7, 8, 9. Declares x only in scope
    ...

for x in [1, 2, 3]: # x is 1, 2, 3. Declares x only in scope
    ...

while loop vary basic loop - loops while condition is true

break and continue statements can be placed within loop. Aborts the innermost loop

if statement will check condition. If True, do first branch, else continue onto other branches.
Standard construct, poerful and simple to understand. Used everywhere

+++ Assignments
Assignments to values are done with x = 5
Cannot assign to a variable which has not been declared.

Variables are declared with x := 5 or x : int = 5. They are scoped to their function. Cannot shadow a binding from within the same function.

def foo():
    if cond:
        x := 5
    else:
        x := 10
    return x # not OK - x is defined within the if's branches, not at the root

def foo():
    x := 0
    if cond:
        x = 5
    else:
        x = 10
    return x # OK - x defined in same scope as return, only assigned to in if statements

def foo():
    x := 0
    if cond:
        x := 5 # Oops! This is an error, and catches a problem caused by probably unintentional shadowing

x := 0
def foo():
    x := 10 # allowed, as x is in a different function. Doesn't change x

x := 0
def foo():
    global x # global statement is required to change x
    x = 10

A function defined inside another function can see variables
defined in the root of the enclosing function, thanks to closures.

def foo() -> int():
    x := 5
    def bar() -> int:
        return x
    return bar
foo()() # 5

Nested functions are nice for teaching basic functional style programming which is becoming more popular

Accessing non root values is not OK

def foo(a : [int]) -> [int()]:
    out : [int()] = []
    for x in a:
        def bar() -> int:
            return x # NOT OK - x is not bound in root of function and thus cannot be seen from enclused function
            # the value x may not have the same value as you are expecting by the time the function is done
            # (only 1 x value is used, so if this was allowed, all functions would return the last value)
        out.push(bar)
    return out

With := statement, type is inferred.

This is only syntax addition to python. Declarations have same semantics at runtime as assignment in python. Only meaning is for Garter.

+++ Classes

Classes are custom types. Defined with

class C:
    field := initialvalue

    def method(self):
        # self is implicitly avaliable here as the instance of C!

Created by calling the name of the class:

x := C()

C itself is a function and a type name.

Methods and fields are accessed with the . operator

Initial values for fields are evaluated once as the class is defined. Can lead occasionally
to confusing results:

class C:
    arr := []
x := C()
x.arr.push(5) #!!! Changing the value of arr for all instances of C() which haven't changed their value of arr!
y := C()
y.arr # Contains 5

This is because of the behavior of python. In python normal way to do this would be:

class C:
    def __init__(self):
        self.arr = []

This violates requirements to not have weird kooky syntax for classes. Seriously considered making
the declaration syntax in classes instead act like __init__ as it is already new syntax. However
the language right now doens't do that.

+++ Expressions

Expressions can also just be written as a statement, and they will be evaluated for their side effects.

++ Programs

A program is just a series of statements. They are executed from top to bottom. It is also possible to interact with a REPL, which evaluates the statements one at a time as they are received.

Functions don't have their bodies typechecked until either:
a) their binding's name is referenced (for example, to be called), or
b) The end of the current program's input is reached.

This allows mutual recursion, for example:

def foo(b: bool):
    if b:
        print('b is true')
    else:
        print('b is false')
        bar() # foo is defined before bar is defined

# if foo was called here, it would be a lookup error, but it isn't called until after so it's OK

def bar():
    foo(True)

foo(False)

++ Modules

Only support for including basic modules provided by language like random, and turtle
Eventual plans to expand to allow user or teacher defined modules (SEE Future work)
Additional screen module provided for Python to enable basic interaction with terminal windows
Would be nice to have simple graphics API for teaching students. Outside the scope of this thesis (SEE future work).
(Hence why copied screen from Turing - existing language for teaching students)


















%The primary goal of the Garter programming language was to provide an
%introductory language for teaching programming. To achieve that goal, we had a
%few requirements:
%
%\begin{enumerate}
%\item The programmer should not be required to use syntax until they understand
%its meaning, yet should be able to start writing code before they have been
%taught about syntax.
%\item The language should not require interaction with language features which
%would not normally be taught at an early level to perform basic tasks.
%\item The language should act as a stepping stone to a production programming
%language, such that eager learners can use what they have learned in class to
%write more complex programs and springboard into learning a language which is
%used by professionals in both Scientific Programming and Application Programming.
%\item Confusing edge cases should be eliminated, such that students and teachers
%are able to safely reason about a program without having to spend time studying
%the language's particulars.
%\item The language should prevent incorrect programs from being written, and
%provide useful explanations as to how to correct the program being run, rather
%than breaking in a confusing manner at runtime.
%\end{enumerate}
%
%\section{Selecting a Base Language}
%
%Requirement 3 meant that we wanted to use an existing programming language as the
%basis for Garter, and then modify it such that it could also satisfy the other
%requirements. By doing that, we would ensure that the language had many of the
%same properties as the production language, and that the developer would be able
%to easily translate the knowledge they had accrued from Garter over to the new
%programming language.
%
%For this purpose, we selected the Python Programming Language as the base language
%\cite{pythonweb}. Python is used within both application programming, especially
%for web development with libraries such as Django \cite{djangoweb}, and
%scientific programmming with powerful high performance math libraries like
%matplotlib and scipy \cite{matplotlibweb} \cite{scipyweb}. This means that
%python has the tools to enable eager programmers to achieve whatever goals they
%have once they leave the safety of Garter.
%
%Python also performs fairly well on requirement 1, with a few exceptions, especially
%related to Python's object model. These design flaws were avoided in
%Section \ref{sec:classes}.
%
%\section{Types}
%
%Python is a dynamic language, which means that minimal analysis of the
%correctness of your program is checked before running the program. Instead,
%incorrect programs fail at runtime through a raised exception, often in a piece
%of code distant from where the actual problem occurred, such as in library code.
%In order to fix this problem, the decision was made to fit Garter with a type
%system to help validate the correctness of programs written in it. 
%
%It was important to select a type system which fit the requirements of a
%learning programming language. One of python's advantages is its dynamic type
%system. Because Python is dynamically typed, functions are extremely generic,
%and only require their parameters to have the methods and properties which are
%actually accessed in the execution of the function. Unfortunately, this can lead
%to hard-to-find problems due to how the constraints are only checked at runtime.
%
%Type systems are largely split into two groups: structural and nominal type systems.
%While a structural type system where the type of a value is defined by its properties
%is closer to python's original reasoning about types, the decision was made to use
%a nominal type system for Garter.
%
%The first reason for selecting to use a nominal type system, where types are
%identified by a name, rather than their properties, is that it can help produce
%better error messages. As a name is avaliable whenever a type is being referred
%to, it is easy to point at what the problem is, for example, by saying
%``Expected an int, instead found a str'' instead of ``Expected a type supporting
%the `/' (division) operation''.
%
%The other reason is related to requirement 3. Of the top 10 languages on the
%popular open source code hosting website GitHub as of August 19,
%2015 \cite{githublangs}, 4 are dynamically typed (JavaScript, Ruby, PHP, and
%Python), 4 are nominally typed (Java, C++, C#, and C), and 2 are not
%general-purpose programming languages (CSS and HTML).
%
%For that reason, we felt that a structural type system was ill-suited to
%introducing people to programming and programming languages.
%
%The decision was also made that we didn't want a highly extensible type system.
%Garter is not intended to be a language to write extensible and re-usable
%libraries in. For that reason, no implementation of generic templated types, or
%functions was implemented, instead Python's built-in list and dictionary types
%were given special status and syntax within Garter, namely \code{[T]} for list
%types, \code{\{K: V\}} for dictionary types, and \code{R(Args, ...)} for function
%types.
%
%The decision to keep the \code{float} and \code{int} types seperate was made due
%to arrays. In Python, it is legal to perform any operation which you could perform
%on a \code{float} with an \code{int}, however the reverse is not true. Namely,
%indexing into an array with a \code{float}, even if it has an integral value,
%is illegal and will cause a runtime exception. As we are trying to avoid runtime
%exceptions, the decision was made to keep the types distinct, but allow for
%implicit $\code{int} \rarrow \code{float}$ conversions.
%
%\section{Syntax}
%
%In accordance with the requirement to provide an easy transition from Garter to
%Python, we wanted to avoid syntax extensions above the base Python syntax as
%much as possible. Instead, we hoped that by restricting the set of legal Python
%programs we could create a new language which shared the same syntax, and could
%simply be extended back to standard python.
%
%Python 3, thanks to PEP 3107, already has support for function annotation
%syntax \cite{pythonfuncannot}. We decided to use this syntax as the syntax
%for Garter function type annotations.
%
%However, we did need to add an additional syntax. In python, there is no declaration
%statement, instead, the first assignment to a variable within a function declares
%that variable within the function's scope. So the code:
%
%\begin{lstlisting}[language=Python]
%x = 'bar'
%def f():
%    x = 'foo'
%f()
%print(x)
%\end{lstlisting}
%
%will print the value \code{'bar'} instead of \code{'foo'}. Instead, in order to
%mutate the global \code{x} value, the statement \code{global x} would have to be
%written in the function \code{f}.
%
%We wanted to catch this type of error in Garter, which means that we wanted to
%seperate the concepts of assigning to a variable and declaring a variable. To
%do this, we introduced additional syntax: \code{x := y} and \code{x : type = y}.
%These expressions have the same runtime semantic meaning as a python assignment
%statement, however they have different meaning to the semantic analysis phase.
%
%In Garter, an assignment can only occur on a variable which has been previously
%declared within the same function, either through \code{:=} or \code{global}.
%This way, attempts to mutate an outer scope without the use of \code{global} are
%caught before execution time.
%
%It also has the benifit of greatly reducing the need for dependence on type inference.
%The \code{:=} statement declares a variable, using type inference to determine its
%type. Unfortunately, there are some times when the type of the right hand side
%of the declaration cannot be determined, for example \code{x := []}. It is known that
%\code{x} is a list, but a list of what? In a situation like that, the other syntax:
%\code{x : type = y} can be used to clarify which particular type should be used
%for \code{x}.
%
%This problem could also be solved by a more powerful type inference system, such
%as the hindley-damas-milner type inference scheme \cite{hm-types-damas}.
%However, such compexity is likely unnecessary for a basic program, and could
%cause confusion when a change in code in one location causes type errors to
%appear at another distinct place in the code due to type inference changing.
%
%No other syntactic additions were made to the language, however Garter's set
%of legal syntax was greatly reduced from the syntax allowed by Python. Garter's
%complete syntax can be found in the appendix.
%
%\section{Functions}
%
%Functions in Garter are defined in the same way that they are defined in Python,
%with the \code{def} statement. We decided to make function bindings created this
%way immutable, such that programmers can't accidentally define multiple
%functions with the same name. Like in Python, we decided to make functions
%objects which can be passed around and assigned to variables. This should enable
%for teachers to introduce some functional aspects to their introductory
%programming courses while using the Garter programming language, without greatly
%increasing the complexity of the language. The fact that a function is an object
%is one which programmers who are not told are unlikely to come across.
%
%Due to the ability for functions to be passed around as objects, the decision was
%also made to allow functions to be defined within other functions. These functions
%are able to access the enclosing function scope of the function they are called in,
%but not the lexical scopes, as those may have values which are no longer valid.
%
%\section{Scopes}
%
%Python has an interesting set of rules for how variables are scoped, due to its
%dynamic nature. Basically, everything in python is scoped at the function level,
%and every variable defined in the function is visible everywhere else in the
%function. That means that code like the following is totally fine:
%
%\begin{lstlisting}[language=Python]
%def foo():
%    if x:
%        y = 10
%
%    if x:
%        return y
%    return 30
%\end{lstlisting}
%
%This is because the presence of an assignment to a variable named \code{y} in the
%function caused the variable \code{y} to be considered function-local. If \code{y}
%was accessed before it had been given a variable, an runtime exception would be
%raised.
%
%Instead, in Garter, the decision was made to make variable scoping lexical. Thus
%whatever lexical scope the declaration (\code{:=}) statement was written in, the
%variable would only be visible in enclosed scopes.
%
%In garter, the decision was made to use lexical scoping, while adding additional
%rules. Namely, it is not possible to shadow a local variable with a new declaration
%in Garter, as in Python the shadowed variable would actually be clobbered by the
%assignment statement.
%
%More details on the scoping rules in Garter can be found in the specification
%(Section \ref{sec:defs_and_scoping}).
%
%\section{REPL Friendliness}
%
%One of the big advantages of a system like Python is its friendliness to
%Read-Eval-Print-Loops (REPLs). Decisions about how scoping and variable
%declarations work in Garter were made in order to make the language more
%friendly to REPLs. For example, in a REPL, you may want to re-define a function
%you previously defined in order to see how the code will act with the new
%function definition. In Python, that is always allowed, because there is no type
%safety, while in Garter, the decision was made to only allow re-declaration in a
%REPL context, when the type of the new function is identical to the type of the
%previous one (so as to not break previously written logic).
%
%This allows for easier experimentation with REPLs in a strongly typed language,
%while making the system more approachable.


