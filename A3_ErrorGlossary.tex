% Appendix 3
\glsresetall % reset the glossary to expand acronyms again
\chapter{Error Glossary}\label{ch:ErrorGlossary}

The error messages are an important part of Garter. Garter programs will be
written by inexperienced programmers, and the error messages which occur when
they write incorrect programs will have to be useful and provide a mechanism for
correcting their errors.

Sometimes, an error will be connected with 'notes', which are intended to
provide additional information for the coder, directing them towards advice. For
this reason, the error messages will occasionally be long, spanning multiple
lines. To avoid scaring away developers with pages of errors, only one error
should be shown at a time.

Only validation-time errors are described here. 

\section{Operator Type Mismatch}
\begin{lstlisting}[breaklines]
Line X, Column X
    ....
     ^
OperatorTypeMismatch:
The operands to operator '+' (int + str), are invalid.
\end{lstlisting}

This error appears when a user attempts to use a unary or binary operator on two types
which are not supported.

If the operator is '+', and at least one of the operands is \code{str}, then we
can predict that the user meant to perform string concatenation, and provide
a fixit note.

\begin{lstlisting}[breaklines]
NOTE: Can only concatenate str and str. Consider casting the left argument
to a str for concatenation with 'str(...)'. 
\end{lstlisting}

\section{Parameter Type Mismatch}
\begin{lstlisting}[breaklines]
Line X, Column X
    ....
     ^
ParameterTypeMismatch:
The 3rd parameter to this function is invalid. Expected int, instead found bool.

NOTE: Function expects parameters (int, int, float)
\end{lstlisting}

This error appears when a user attempts to call a function, either internal or user-defined,
with the incorrect type parameter.

The expected parameters are listed to help the user correct their logic quickly.

\section{Parameter Count Mismatch}
\begin{lstlisting}[breaklines]
Line X, Column X
    ....
     ^
ParameterCountMismatch:
This function expects 3 parameters, instead found 4 parameters.

NOTE: Function expects parameters (int, int, float)
\end{lstlisting}

This error appears when a user attempts to call a function, either internal or user-defined,
with the incorrect number of arguments.

The expected parameters are listed to help the user correct their logic quickly.

\section{No Such Attribute}
\begin{lstlisting}[breaklines]
Line X, Column X
    ....
     ^
NoSuchAttribute:
This object of type 'str' has no attribute named 'jion'.

NOTE: 'str' has attributes: 'join'.
\end{lstlisting}

This error appears when a user attempts to access an attribute of an object
which is not available. The list of available attributes is listed to help them
find the attribute they ment to access.

\section{Attribute Already Defined}
\begin{lstlisting}[breaklines]
Line X, Column X
    ....
     ^
AttributeAlreadyDefined:
This class has a duplicate definition of the attribute 'foo'.

NOTE: The previous definition of this attribute is here:
Line Y, Column Y
    ....
     ^
\end{lstlisting}

This error occurs when a user attempts to define an attribute on an object which
has previously been defined. It identifies the location of the previous
definition, so that the user can quickly correct the offender.


\section{Variable Already Defined}
\begin{lstlisting}[breaklines]
Line X, Column X
    ....
     ^
VariableAlreadyDefined:
The variable 'foo' has already been defined in this scope.

NOTE: The previous definition of this variable is here:
Line Y, Column Y
    ....
     ^
\end{lstlisting}

This error occurs when a user attempts to define an variable in a scope where it
has previously been defined. It identifies the location of the previous
definition, so that the user can quickly correct the offender.

\section{Invalid Variable}
\begin{lstlisting}[breaklines]
Line X, Column X
    ....
     ^
InvalidVariable:
The variable 'foo' may not have a valid value when it is accessed here.
\end{lstlisting}

This error occurs when a user attempts to access a variable which may not be
valid when it is accessed. This might happen if, for example, a function is
declared in a loop, and the function attempts to access the iterator value.

\section{Incomplete Type}
\begin{lstlisting}[breaklines]
Line X, Column X
    ....
     ^
IncompleteType:
Unable to infer the type of this value. Try annotating this declaration with the desired type.
\end{lstlisting}

This error occurs when the user attempts to assign an unwritable type to a variable or field.
For example, the declaration \code{x := []}, for which the item type of the expression cannot be determined.

If it is a simple error like \code{x := []}, \code{x := \{\}}, or \code{x := None} then we also add one of these notes:
\begin{lstlisting}[breaklines]
NOTE: Annotate this declaration to specify the item type of the list
NOTE: Annotate this declaration to specify the key and value types of the dict
NOTE: Annotate this declaration to specify the class type
\end{lstlisting}

\section{Invalid Typecast Source}
\begin{lstlisting}[breaklines]
Line X, Column X
    ....
     ^
InvalidTypecastSource:
Cannot cast to type 'float' from type 'bool'.
\end{lstlisting}

This error occurs when a user attempts to typecast from an invalid type. It identifies
both of the types.


\section{Mismatched Branch Types}
\begin{lstlisting}[breaklines]
Line X, Column X
    ....
     ^
MismatchedBranchTypes:
The types of the branches of the if expression must match.
Instead found 'int' and 'bool'
\end{lstlisting}

This error occurs when an if expression with mismatched types on the then and
else branches is written.

\section{Not In Loop}
\begin{lstlisting}[breaklines]
Line X, Column X
    ....
     ^
NotInLoop:
The 'break' and 'continue' statements may only be written in loops.
\end{lstlisting}

This error occurs when the user writes a \code{break} or \code{continue} statement
outside of a loop.

\section{Return Outside Function}
\begin{lstlisting}[breaklines]
Line X, Column X
    ....
     ^
ReturnOutsideFunction:
The 'return' statement may only be written within functions.
\end{lstlisting}

This error occurs when the user writes a \code{return} statement outside of a function or method.

\section{Invalid Len Argument}
\begin{lstlisting}[breaklines]
Line X, Column X
    ....
     ^
InvalidLenArgument:
The len expression only accepts lists, dicts, and strs. Instead found a 'bool'.
\end{lstlisting}

This error occurs when an invalid argument is passed to the len magic function.

\section{Invalid Return Type}
\begin{lstlisting}[breaklines]
Line X, Column X
    ....
     ^
InvalidReturnType:
This function must return type 'bool', instead found a 'str'.
\end{lstlisting}

This error occurs when a returns statement in a function has an invalid type.

If the function shouldn't return any type, the error will instead be:

\begin{lstlisting}[breaklines]
This function doesn't return a type, instead found a 'str'.
\end{lstlisting}

If the function should return a type but there is not one written

\begin{lstlisting}[breaklines]
This function must return type 'bool', instead found an argument-less return statement
\end{lstlisting}

\section{Invalid Conditional}
\begin{lstlisting}[breaklines]
Line X, Column X
    ....
     ^
InvalidConditional:
The type of the conditional in an if or while must be 'bool'. Instead found 'int'
\end{lstlisting}

This error occurs when a non-\code{bool} value is used as the conditional in an
\code{if} or \code{while}.

\section{Mismatched List Type}
\begin{lstlisting}[breaklines]
Line X, Column X
    ....
     ^
MismatchedListType:
The type of elements in a list must be consisitent.
The first element in this list literal is 'bool', while the 3rd element is 'str'
\end{lstlisting}

This error occurs when a list literal is used with inconsistent item types.

\section{Mismatched Dict Type}
\begin{lstlisting}[breaklines]
Line X, Column X
    ....
     ^
MismatchedDictType:
The type of keys and values in a dict must be consisitent.
The first key in this dict literal is 'bool', while the 3rd key is 'str'
\end{lstlisting}

This error occurs when a list literal is used with inconsistent key/value types.

\section{Invalid Print Line End}
\begin{lstlisting}[breaklines]
Line X, Column X
    ....
     ^
InvalidPrintLineEnd:
The 'end' named parameter to the print statement must be 'str', instead found 'bool'
\end{lstlisting}

This error occurs when the \code{end} named parameter to the \code{print} function is
the incorrect type.

\section{Invalid Index Type}
\begin{lstlisting}[breaklines]
Line X, Column X
    ....
     ^
InvalidIndexType:
Only 'int' may be used to index into '[bool]'. Instead found 'float'
\end{lstlisting}

This error occurs when the incorrect type is used for indexing into \code{list},
\code{dict} or \code{str} types with the subscription syntax.

\section{Unsupported Index}
\begin{lstlisting}[breaklines]
Line X, Column X
    ....
     ^
UnsupportedIndex:
Cannot index into type 'bool'.
\end{lstlisting}

This error occurs when the indexing form of the subscription syntax is used on a
type which doesn't support it.

\section{Unsupported Slice}
\begin{lstlisting}[breaklines]
Line X, Column X
    ....
     ^
UnsupportedSlice:
Cannot slice into type 'bool'.
\end{lstlisting}

This error occurs when the slicing form of the subscription syntax is used on a
type which doesn't support it.

\section{Invalid Assign Target}
\begin{lstlisting}[breaklines]
Line X, Column X
    ....
     ^
InvalidAssignTarget:
This expression is immutable, and thus cannot be assigned to.
\end{lstlisting}

This error occurs when the user attempts to assign to an expression which isn't
a valid assignment target.
