%*******************************************************************************
% Preamble
%*******************************************************************************
% Instead of inserting \usepackage and defined commands here, they are in a separate file
% Queen's Thesis Format
% (Borrowed from Dean Jin's BigDis.tex file, then heavily modified :)

% Originally:
% Michelle L. Crane, Queen's University, 2003

% Then:
% Andrew W L Dickinson, Queen's University, 2011

% This revision:
% Kevin Hughes, Queen's University, 2013

%*******************************************************************************
% DOCUMENT STYLE
%*******************************************************************************
\documentclass[12pt]{report}
%-------------------------------------------------------------------------------------------------------------
\usepackage{quthesis}        % the Queen's University dissertation style file

%I don't even use the fancyheadings - it looks nice enough without it
%\usepackage{fancyheadings}  % doesn't seem to change the headings at all!
%*******************************************************************************


%*******************************************************************************
% SPACING
%*******************************************************************************
\usepackage{setspace}        % for use of \singlespacing and \doublespacing
%*******************************************************************************


%*******************************************************************************
% HEADINGS
%*******************************************************************************

% This changes the headings go that they are prettier, this can be commented out for traditional headings.
\usepackage{sectsty}
\allsectionsfont{\bfseries}% set all the section font to bfseries
\chapterfont{\centering\Large} % set the sizes of chapters, sections ...
\sectionfont{\normalsize}
\subsectionfont{\normalsize}

% for formatting Table of Contents entry, example: Chapter 1 Introduction
\usepackage[subfigure]{tocloft}
\usepackage{tocloft}
\renewcommand{\cftchappresnum}{Chapter }
\renewcommand{\cftchapaftersnum}{:}
\renewcommand{\cftchapnumwidth}{7em}

% for formatting Table of Contents entry for Appendix, example: Appendix 1: Stuff
\newcommand*\updatechaptername{%
   \addtocontents{toc}{\protect\renewcommand*\protect\cftchappresnum{Appendix }}
}

%*******************************************************************************
% FOOTNOTES
%*******************************************************************************

\interfootnotelinepenalty=10000 % This line stops footnotes from splitting onto two pages.

%*******************************************************************************
% VERBATIM
%*******************************************************************************
\usepackage{moreverb}        % Using this package to get better control of the
                             % verbatim environment, mostly for the use of the
                             % listing environment which puts line number
                             % beside each line.  Note that there has to be a number
                             % in each set of brackets, i.e., \begin{listing}[1]{1}.
                             % PDF info file is "The moreverb package" by
                             % Robin Fairbairns (rf@cl.cam.ac.uk) after
                             % Angus Duggan, Rainer Schopf and Victor Eijkhout, 2000/06/29.
%-------------------------------------------------------------------------------
%\usepackage{verbatim}        % allows the use of \begin{comment} and \end{comment}
                             % as well as \verbatiminput{file}
                             % Note:  when using verbatim to input from a text file,
                             % such as a specification or code, use \begin{singlespacing}
                             % and \end{singlespacing}.  Also, tabs are not read
                             % properly, so the input file must only use spaces.

%                             \begin{comment}
%                             Can also use the verbatim package for
%                             comments like this...
%                             \end{comment}
%*******************************************************************************


%*******************************************************************************
% GLOSSARY
%*******************************************************************************
\usepackage[nonumberlist]{glossaries}
\makeglossaries
\glossarystyle{list}
%*******************************************************************************


%*******************************************************************************
% LIST OF SYMBOLS
%*******************************************************************************
\usepackage{nomencl}
\makenomenclature
%*******************************************************************************


%*******************************************************************************
% INDEX
% Also possible to make an index; didn't use for my thesis.
%*******************************************************************************
\usepackage{makeidx}         % to make the index
\makeindex
%*******************************************************************************


%*******************************************************************************
% MATH STUFF
%*******************************************************************************
%-------------------------------------------------------------------------------
\usepackage{amsmath}         % to make nice equations
%-------------------------------------------------------------------------------
\usepackage{amsthm}          % to make nice theorem, i.e., definition

% Using the amsthm package, define a new theorem environment for my
% definition.  * means don't number it.
\newtheorem*{definition}{Definition}
%-------------------------------------------------------------------------------
\usepackage{cases}           % to make numbered cases (equations)
%-------------------------------------------------------------------------------
\usepackage{calc}            % Used with the Ventry environment defined below.
%*******************************************************************************


%*******************************************************************************
% FLOATS AND FIGURES
%*******************************************************************************
\usepackage{graphicx}        % for graphic images (use \includegraphics[...]{file.eps})
%-------------------------------------------------------------------------------
\usepackage{subfigure}       % for subfigures (figures within figures)
%-------------------------------------------------------------------------------
\usepackage{boxedminipage}   % to make boxed minipages, i.e., boxes around figures
%-------------------------------------------------------------------------------
\usepackage{rotate}          % for use of \begin{sideways} and \end{sideways}
%-------------------------------------------------------------------------------
\usepackage{listings}        % for use of printing code blocks
% Define how I would like the ``listings'' to look.
\lstset{basicstyle=\footnotesize,, numbers=left, numberstyle=\small, stepnumber=1, numbersep=5pt, showspaces=false, lineskip=-1pt}
%-------------------------------------------------------------------------------
\usepackage{float}           % Using this package to get better control of my floats
                             % including the ability to define new float types for
                             % my specification and code listings.
                             % DVI info file is "An Improved Environment for Floats"
                             % by Anselm Lingnau, lingnau@tm.informatik.uni=frankfurt.de
                             % 1995/03/29.

% Define new float styles here
% Ruled style for examples
%\floatstyle{ruled}
%\newfloat{Example}{h}{lop}[chapter]

% Style of float used for code listings
\floatstyle{ruled}
\newfloat{Listing}{H}{lis}[chapter]

                             % Note:  The listings don't have space between the chapters, unlike
                             % the standard list of tables etc.  At the end, copy the spacing
                             % commands from the .toc file and insert into the .lis file.  Then,
                             % DO NOT LATEX it again, simply go to the DVI viewer!

\usepackage{placeins} % for \FloatBarrier which stops floats from crossing a point
%*******************************************************************************
% TABLES
%*******************************************************************************
\usepackage{tabularx}        % Package used to make variable width-columns, i.e.,
                             % column widths are changed to fit the maximum width
                             % and text is wrapped nicely.

\usepackage{threeparttable}
%*******************************************************************************
% CAPTIONS
%*******************************************************************************
\usepackage[hang]{caption}   % Package used to make my captions 'hang', i.e., wrap
                             % around, but not under the name of the caption.
%-------------------------------------------------------------------------------------------------------------
% Find that the captions are too far from my verbatim figures, but if
% I change it to 0, then the captions are too close for my other types
% of figures.  Maybe set each one separately?
%\setlength{\abovecaptionskip}{1ex}

%\setlength{\textfloatsep}{1ex plus1pt minus1pt}

%\setlength{\intextsep}{1ex plus1pt minus1pt}

%\setlength{\floatsep}{1ex plus1pt minus1pt}
%*******************************************************************************


%*******************************************************************************
% MISCELLANEOUS
%*******************************************************************************
\usepackage{layout}          % useful for determining the margins of a document
                             % use with \layout command
%-------------------------------------------------------------------------------
\usepackage{changebar}       % Way of indicating modifications by putting bars in the
                             % margin.  Read about it in "The Latex Companion".
%-------------------------------------------------------------------------------
%\usepackage{ccfonts,eulervm} 	% fonts I like from Knuth's "Concrete Mathematics"
%\usepackage[T1]{fontenc}
%*******************************************************************************

%*******************************************************************************
% REFERENCES ETC.
%*******************************************************************************
\usepackage{varioref}        % Better page references, e.g., "on preceding page", etc.
                             % \vref{key} Create an enhanced reference.
                             % \vpageref[text]{key} Create an enhanced page reference.
                             % \vrefrange{key}{key} Create an enhanced range of references.
                             % \vpagerefrange[text]{key}{key} Create an enhanced range of page references.
                             % Note: doesn't really work for consecutive pages.

% Renewing the text for before and after, because I don't like the default flip-flopping one.
% And 'on the page before' sounds dumb!

\renewcommand{\reftextafter}{on the next page}
\renewcommand{\reftextbefore}{on the previous page}
%-------------------------------------------------------------------------------
\usepackage{url}             % for use of \url - pretty web addresses
\usepackage{fancyhdr}
\usepackage{cite}

%*******************************************************************************
% HYPERLINKS (must be last)
%*******************************************************************************

% Uncomment these next two lines for linkback to citation pages in biblio
% \renewcommand*\backref[1]{\ifx#1\relax \else \linebreak Cited on page(s): #1. \fi}

\usepackage[bookmarks,pdfauthor={Michael Layzell}, pdftitle={Garter: A Small to Medium Sized Snake}]{hyperref}

\hypersetup{colorlinks=true,  % Change links to being coloured text, no boxes
		linkcolor=blue,
}
                             % Neat package to turn href, ref, cite, gloss entries
                             % into hyperlinks in the dvi file.
                             % Make sure this is the last package loaded.
                             % Use with dvips option to get hyperlinks to work in ps and pdf
                             % files.  Unfortunately, then they don't work in the dvi file!
                             % Use without the dvips option to get the links to work in the dvi file.

                             % Note:  \floatstyle{ruled} don't work properly; so change to plain.
                             % Not as pretty, but functional...
                             % The bookmarks option sets up proper bookmarks in the pdf file :)

% Need this command to allow hyperref to play nicely with gloss; otherwise
% almost every \gloss will cause an error...
%\renewcommand{\glosslinkborder}{0 0 0}
%*******************************************************************************


%*******************************************************************************
% MISCELLANEOUS COMMANDS AND ENVIRONMENTS
%*******************************************************************************
% Use this command to show more table of contents - used when playing
% with the draft outline
% I think it should be about 2???
\setcounter{tocdepth}{2}
%*******************************************************************************
% Environment definition I found in the "The Latex Companion".  Used to
% create a list environment where the indenting is the same for all of the
% entries, regardless of their length.  Note:  must \usepackage{calc}.
\newenvironment{Ventry}[1]%
    {\begin{list}{}{\renewcommand{\makelabel}[1]{\textbf{##1}\hfil}%
        \settowidth{\labelwidth}{\textbf{#1:}}%
        \setlength{\leftmargin}{\labelwidth+\labelsep}}}%
    {\end{list}}
%*******************************************************************************

%*******************************************************************************
% MY DEFINED COMMANDS
%*******************************************************************************
% Command that I can use to create notes in the margins;
% adapted from Juergen's META tag
%\newcommand{\meta}[1]{\begin{singlespacing}
%{\marginpar{\emph{\footnotesize Note: #1}}}\end{singlespacing}}
%*******************************************************************************
% Command that I can use to create lined headings
%\newcommand{\heading}[1]{\bigskip \hrule \smallskip \noindent \texttt{#1} \smallskip \hrule}
%*******************************************************************************
% Command that I can use for reading in a file, verbatim, with line
% numbers printed along the left side.  The parameter is the file name.
%\newcommand{\fileinnum}[1]{
%    \begin{singlespacing} {\footnotesize
%    \begin{listinginput}[1]{1}{#1}\end{listinginput}
%    }\end{singlespacing}
%}
%*******************************************************************************
% Command that I can use for reading in a file, verbatim, with NO line
% numbers, but in a smaller font.  The parameter is the file name.
\newcommand{\filein}[1]{
   \begin{singlespacing}{\footnotesize
    \begin{verbatiminput}{#1}\end{verbatiminput}
    }\end{singlespacing}
}
%*******************************************************************************
% Command that I can use for reading in a file, verbatim, with NO line
% numbers, but in a smaller font.  The parameter is the file name.
\newcommand{\fileinsmall}[1]{
    \begin{singlespacing}{\scriptsize
    \begin{verbatiminput}{#1}\end{verbatiminput}
    }\end{singlespacing}
}
%*******************************************************************************
% Dean't 'notesbox' command.  Needs setspace package.
%   Usage: \notesbox{This is a note.}
%%
\usepackage{setspace}
\newcommand{\notesbox}[1]{
     \begin{singlespacing}
      \noindent\begin{boxedminipage}[f]{\textwidth}{\sf{#1}}\end{boxedminipage}
	\vspace{-24pt}
     \end{singlespacing}
}


% Change the author name and pdf title in the preamble line 235

%*******************************************************************************
% DOCUMENT
%*******************************************************************************

\begin{document}

%*******************************************************************************
% TITLE
%*******************************************************************************

\title{Garter: A Harmless Small to Medium Sized Snake}

\author{Michael Layzell}

\dept{Department of Computer Science}
\degree{Bachelor of Computing}

% ~~~~~~~~~~~~~
%\submitdate{June 2013}
%        - date LaTeX'd if omitted
%\copyrightyear{2013}
%        - year LaTeX'd if omitted

\beforepreface

%*******************************************************************************
% ABSTRACT
%*******************************************************************************

\prefacesection{Abstract}

Languages used for teaching need to be easy to use, help new programmers find
and correct the errors which they make in the program, and are ideally well
equipped for developing actual software, to allow the student to explore outside
of class with their own projects. In this work, we design and implement
a subset with 'training wheels' of the dynamically-typed programming language
Python \cite{pythonweb} for use in teaching, in order to create a language for
learning which better fits those criteria.

Dynamic programming languages such as the Python Programming Language can be
powerful tools for experienced developers due to their high productivity
potential. Unfortunately, when learning, the error messages which they produce
can seem arcane or arbitrary. Garter is a new language, based on the existing
dynamic language Python, which aims to produce high quality error messages, and
provide a comfortable learning environment for new developers, while exposing
them to syntax and semantics which are used in the real world.


%*******************************************************************************
% ACKNOWLEDGEMENTS
%*******************************************************************************

%\phantomsection
%\prefacesection{Acknowledgements}
%Acknowledge some people here

%\clearpage % keep this for proper page numbering!

%*******************************************************************************
% STATEMENT OF ORIGINALITY (required CHEM, CISC, GEOL, MATH, PHYS (Ph.D. only))
%*******************************************************************************

%\phantomsection
%Statement of Originality goes here if required
%\clearpage

%*******************************************************************************
% Table of Contents
%*******************************************************************************
% The Table of Contents and the following sections up until the main chapters are generated
% by code in the quthesis.sty file particularly \def\afterpreface... around line 248

%*******************************************************************************
% Glossary
%*******************************************************************************
% make sure Glossary Packages etc. are uncommented in 0_Preamble.tex
%\glosspagetrue
%\newacronym{AI}{AI}{Artificial Intelligence}

\newglossaryentry{OpenCV}{
	name={OpenCV},
	description={Open source Computer Vision library for C++\cite{opencv}}
}

\glsaddall


%*******************************************************************************
% List of Symbols
%*******************************************************************************
% make sure List of Symbols Packages etc. are uncommented in 0_Preamble.tex
%\symbolspagetrue
\symbolspagefalse

%*******************************************************************************
% List of Figures
%*******************************************************************************
%\figurespagetrue
\figurespagefalse

%*******************************************************************************
% List of Code Listings
%*******************************************************************************
%\listingspagetrue
\listingspagefalse

%*******************************************************************************
% List of Tables
%*******************************************************************************
%\tablespagetrue
\tablespagefalse

%*******************************************************************************
% CHAPTERS
%*******************************************************************************
\singlespacing \afterpreface \doublespacing

% This command can be used to view the page layout for this document
% \layout

% Here, I'm tweaking how much space is put above and below floats.
% Comment out if you want the *purest* latex spacing.
%\setlength{\abovedisplayskip}{3pt plus1pt minus1pt}
%\setlength{\abovedisplayshortskip}{3pt plus1pt minus1pt}
%\setlength{\belowdisplayskip}{3pt plus1pt minus1pt}
%\setlength{\belowdisplayshortskip}{3pt plus1pt minus1pt}


% Include chapters - kept separated to make editing easier.
% Chapter 1
\glsresetall % reset the glossary to expand acronyms again
\chapter{Introduction}\label{ch:Introduction}

% Motivation
\section{Motivation}

Modern software development has moved toward the increased usage of dynamic
programming languages. These languages do not contain a complete semantic
analysis phase during compilation, and instead handle and report any type errors
at run-time, as incorrect code is executed. This serves as an advantage for
programmers due to the ability to easily write generic code (as all code is, by
default, generic over all compatible types), as well as not have to design a
complete data layout before beginning to write code. This allows for projects to
be written quickly and easily. In program contexts where the optimizations only
available to statically typed languages are not necessary, these can be
significant benefits.

Unfortunately, this dynamic execution doesn't only have run-time costs. It also
affects the ease of understanding program errors, usually quite negatively. As
an example, the same basic programming errors have been introduced into two
programs: we will compare the error outputs of two similar incorrect problems,
one written in Python, which is a dynamic language, and the other written in
C \cite{ansic}, which is a static language. The python programs are executed
with the cpython 2.7.10 \cite{cpython} interpreter, and the C programs are
compiled using clang \cite{clangweb}, with the \code{-Werror} flag enabled.

\begin{minipage}[t]{0.5\textwidth}
\lstinputlisting[language=Python]{figs/python1.py}
\end{minipage}
\begin{minipage}[t]{0.5\textwidth}
\lstinputlisting[language=C]{figs/c1.c}
\end{minipage}

When the python program on the left is run with cpython, the program begins
executing and reaches line 5 before producing a run-time error: \code{NameError:
name 'xx' is not defined}. In contrast, the C program fails to compile, emitting
the compile-time error \code{error: use of undeclared identifier 'xx'}. If the
python code was to refer to the incorrectly named code in an infrequent code
path, the error could never be caught and reported, even with testing.

\begin{minipage}[t]{0.5\textwidth}
\lstinputlisting[language=Python]{figs/python2.py}
\end{minipage}
\begin{minipage}[t]{0.5\textwidth}
\lstinputlisting[language=C]{figs/c2.c}
\end{minipage}

These programs, much like the earlier programs, also demonstrate a common error:
passing values with incorrect types into a function. In the python code, the
error is reported at run-time on line 3 - \emph{within} the function. This
function could theoretically be located within a library which a developer is
using, meaning that the error is reported at a code location which is unrelated
to the actual programming error. In contrast, The strongly-typed C program
reports the error on line 10, which is the call site where the actual mistake
occurred.


% Problem Overview
\section{Problem Overview}
\label{sec:requirements}

The goal of this thesis was to design, implement, and demonstrate a new teaching
programming language: Garter. Garter is intended to be a language which both
makes teaching programming concepts to new developers easy, and also provides
a smooth path for moving away from the learning environment to implementing and
solving real problems. We consider such a language to have to fit the following
requirements:

\begin{enumerate}
\item Minimal Boilerplate: The programmer should not have to write code which
they to not yet understand simply in order to get the code which they do
understand to work.
\item Stepping Stone to Production: The language should act as a stepping stone
to a production programming language, such that eager learners can use what they
have learned in class to write more complex programs and springboard into
learning a language which is used by professionals in both Scientific
Programming and Application Programming.
\item Minimal Technical Trivia: When teachers use the language to teach, they
should feel like they are teaching programming, rather than teaching a language.
Strange edge cases and specific technical trivia about how features were
designed or implemented should be eliminated.
\item Prevent Incorrect Programs: It's very easy in dynamic languages to write
incorrect programs which fail at runtime with confusing error messages, often
lacking important context. Programs should instead fail as fast as possible with
useful error messages which guide the programmer toward writing the correct
program.
\end{enumerate}

% Thesis Contributions
\section{Thesis Contributions}

The main contributions of this thesis are as follows:

\begin{itemize}
\item We designed a new teaching programming language, Garter, which aims to
bring together the simplicity of other teaching programming languages, like
Turing, and the more modern dynamic programming language Python.
\item We created a prototype implementation of Garter, based on the cpython
runtime, including using the cpython IDE, IDLE, and adapting it to work with
Garter code.
\item We implemented a series of example programs, in Garter, demonstrating
that it can be used to solve simple problems like those solved in classrooms,
and showing how to migrate Turing style programs to Garter.
\end{itemize}


% Thesis Outline
\section{Thesis Outline}
The remainder of this thesis is organized as follows:

\noindent\textbf{Chapter 2, Background:} An exploration of the other languages and tooling which exist in this space

\noindent\textbf{Chapter 3, Design:} A discussion of the design decisions made for the Garter language to make it fit the 4 requirements.

\noindent\textbf{Chapter 4. Prototype Implementation:} An exploration of the prototype implementation of Garter

\noindent\textbf{Chapter 5, Conclusions and Future Work:} Where Garter has succeeded and failed, as well as the path foward to a better educational programming language

% Chapter 2
\glsresetall % reset the glossary to expand acronyms again
\chapter{Background}\label{ch:Background}
Background

% Sections
\section{Examples}
text
\subsection{Sub Section}
text
\subsubsection{Sub Sub Section}
text

% Example Glossary
\noindent \gls{AI}

% Example Reference
\noindent thanks to OpenCV\cite{opencv}

% Example Figure
% reference this figure in the text using \ref{fig.testPlot}
\begin{figure}[!htbp]
	\centering
	\includegraphics[width = 5in]{figs/testPlot.png}
	\caption{Test Plot}
	\label{fig.testPlot}
\end{figure}

% Example Equation
% I recommend using http://www.codecogs.com/latex/eqneditor.php to help make equations
% for tough equations or until you're familiar with LaTeX
% reference in the text using \ref{eqn.lowpassfilter}
\begin{equation}
	\mu_t = \alpha x + (1 - \alpha) \mu_{t-1}
	\label{eqn.lowpassfilter}
\end{equation}

% Example Nomenclature
\nomenclature{$\mu$}{Average}

% Example Code Listing
% reference in the text using \ref{ls.testPlot}
\lstset{language=python}
\lstset{tabsize=4}
\lstset{commentstyle=\color{blue}}
\lstset{frame=single}
\lstset{label = ls.testPlot}
\lstset{caption = Test Plot Code }
\lstinputlisting[float=!htbp]{figs/testPlot.py}

% Example Table
% I recommend using http://truben.no/latex/table/ to help with making tables
% reference in the text using \ref{tab.testTable}
\begin{table}[!htbp]
    \centering
    \begin{tabular}{|l|l|l|} % options are l,c,r (left, center, right)
    \hline
    ~ & ~ & ~ \\
    \hline
    ~ & ~ & ~ \\
    \hline
    ~ & ~ & ~ \\
    \hline
    ~ & ~ & ~ \\
    \hline
    \end{tabular}
    \caption{Test Table}
    \label{tab.testTable}
\end{table}

% Chapter 3
\glsresetall % reset the glossary to expand acronyms again
\chapter{Design}\label{ch:Design}

For this thesis, we wanted to explore how we could create a better programming
language for teaching new programmers. Of the requirements we identified in
Section \ref{sec:requirements}, Existing programming languages for teaching such
as Turing \cite{turingpaper} for the most part fulfil the requirements of being
simple to start in, having few confusing edge cases, and restricting the types
of programs which can be written to those which are correct and don't fail at
runtime. Unfortunately, many of them fail when it comes to being a good stepping
stone for students into producting programming languages which people in the
industry use to solve real problems. It is important to reduce the barriers
between the language which people learn in school and solving real-world
problems, as one of the best ways to learn programming is to get engaged in
trying to solve a problem you are facing, which is much easier when you have
access to the libraries and resources provided to and by production programmers.

To satisfy this requirement, Garter was designed as a safe subset of the Python
programming language \cite{pythonweb}. Python is a popular programming language
for modern development, ranking as the 5th most popular language on the GitHub
code sharing platform in August 2015 \cite{githublangs}. The goal was to make
the two languages so similar that it would be trivial for an eager student to
transition their knowledge from writing Garter programs into solving real
problems in Python using the substantial suite of libraries and tools which
Python provides.

In this chapter the design decisions which were made in establishing the Garter
subset of Python are described.

\section{Making Python Type Safe}

One of the desirable properties of a teaching programming language is that it
prevents new developers from writing incorrect programs, and guides them away
from making simple mistakes, especially when these problems only occur at
runtime, and can thus be vary hard to diagnose.

\begin{lstlisting}[language=Python]
x := 5
s := 'x = ' + x
\end{lstlisting}

The above program is written in Garter, and would produce a validation error,
reporting an \code{OperatorTypeMismatch}, \code{The operands to operator '+'
(str + int), are invalid.} The equivalent program in Python would not fail until
runtime, when a \code{TypeError} would be raised. If this code was in an
infrequently traversed code path, such as a failure path, it could remain
undetected, meaning that the program is subtly wrong. In the development of the
Garter prototype, I often accidentally performed this very error within the
error handling code, and didn't catch them until I wrote the test cases for
validation errors.

In addition, we want to prevent code which, while not technically incorrect,
is likely to cause problems in the future. For example, Python allows for
hetrogenous arrays, consisting of values of varying types. This means it's
possible to write code such as:

\begin{lstlisting}[language=Python]
arr := [1, 2, 3, 4, 5, 6, 7, 8, 9]
arr.push(input('Enter a number: '))
\end{lstlisting}

While this code is technically correct in Python, and would execute correctly as
a Python program, future code which expects the elements in \code{arr} to
consist only numbers would fail, as one of the elements is a string. In Garter,
this would produce a \code{ParameterTypeMismatch} validation error, as arrays
are required to be homogenous, which is usually what is desired, such that
individual elements can be treated uniformly.

This means that if you write code like the following in Garter, and it
validates, you can always depend on it working, and there not being any strings
which cause runtime failures far from the source of the problem.

\begin{lstlisting}[language=Python]
sum := 0.0
for elt in arr:
    sum += elt
\end{lstlisting}

To fix this problem, we adapted a simple type system based languages such as
Pascal and Turing. The type system modified to use Python's basic types, such
as \code{int}, \code{float}, \code{bool}, \code{str}, lists (\code{[T]}), dicts
(\code{\{K:V\}}), functions (\code{R(A, R, G...)}), and user defined classes.
This type system limits what values are allowed to exist in the language, such
that only the common use cases, such as homogenous arrays and dictionaries, are
permitted. This means that new developers will be guided away from writing code
which accidentally creates one of these non-homogenous data types or variables,
which is usually an error, even in real Python code.

In Garter we also decided against implementing generic or templated types. The
dict and list types are instead given special status. This was because we
decided that the complexities and design tradeoffs which are involved in this
complex feature were unnecessary, especially for early learning. Instead, the
use of the built-in types is heavily encouraged. Some programming languages
which are used in the industry, such as the Go Programming Language also lack a
generics system, which helped us make this decision \cite{golangweb}.

The possibility of merging the \code{int} and \code{float} types into a single
\code{num} type was considered, however this would have confusing properties
when it comes to array indexing, and what values are OK for using there. In
Python, only \code{int} values can be used for indexing into an array. This
behavior of treating the value \code{2} differently than \code{2.0} would be
very confusing to a new programmer, so the decision was made to formalize the
difference to avoid confusion and common programming errors, especially related
to division of integers.

\begin{lstlisting}[language=Python]
arr[2] # OK
arr[2.0] # Not OK
arr[4/2] # Not OK
arr[4//2] # OK
\end{lstlisting}

\section{Objects}
Data in Garter are represented by Objects. 

\subsection{Value passing style}
Object types can be split into two categories. The primitive
types: \code{int}, \code{float}, \code{str}, \code{bool} and functions have
immutable values, which means that they act as though they are passed by value.

\begin{lstlisting}[language=Python]
x := 5
y := x
x = 10
print(y) # 5
\end{lstlisting}

In contrast, the other types, such as \code{[T]}, \code{\{K: V\}} and user defined types
are not immutable, and act as though they are passed as a reference to a shared object.

\begin{lstlisting}[language=Python]
arr := [1, 2, 3]
arr2 := arr
arr.push(4)
print(arr2) # [1, 2, 3, 4]
\end{lstlisting}

If a new copy of a list or dictionary is required, it can be copied with
the \code{.copy()} method.

\begin{lstlisting}[language=Python]
arr := [1, 2, 3]
arr2 := arr.copy()
arr.push(4)
print(arr2) # [1, 2, 3]
\end{lstlisting}

This behavior is inherited from Python, and directly mimics the behavior used in
many programming languages, where simple types are immutable but complex ones
are mutable and shared by reference.

\subsection{Functions and Methods}

Functions in Garter are implemented as objects, which can be passed around and
have the type \code{R(A, R, G...)}, where \code{R} is the return type, and
\code{A}, \code{R}, and \code{G} are the argument types. The following is the
representation of the types of some functions

\begin{lstlisting}[language=Python]
def foo(): ...                         # None()
def bar(x : int) -> int: ...           # int(int)
def baz() -> int: ...                  # int()
def quxx(x : int, y : int) -> int: ... # int(int, int)
\end{lstlisting}

This decision was made in order to enable teachers to teach some basic
functional programming styles, which are becoming more popular in modern
programming due to the popularity of web programming with languages such as
JavaScript which often use callbacks and other functions as values.

Methods are represented as immutable function attributes on objects, except with
the an implicit \code{self} argument.

\begin{lstlisting}[language=Python]
class C:
    s := 'hi'
    def f(self): print(self.s)
x := C()
y : None() = x.f # Valid
y() # prints hi
\end{lstlisting}

This conforms with Python's representation of Methods, and avoids confusion
around methods, especially when using them in more functional programming styles
which some teachers may choose to use to teach their students.

\section{Expressions}

Expressions are based on the expression forms from Python. Like in Python, all
expressions are evaluated eagerly, which both conforms with the current popular
design for production languages, and simplifies reasoning about when code is
executed. 

\subsection{Literals}

Garter supports the following literals:
\begin{itemize}
\item list literals (e.g. \code{[1, 2, 3]})
\item dict literals (e.g. \code{\{'a': 1, 'b': 2\}})
\item numeric literals (e.g. \code{5}, and \code{5.5})
\item bool literals (\code{True}, and \code{False})
\item str literals (e.g. \code{'foo'}, and \code{"bar"})
\end{itemize}

The list and dictionary literals especially enable much shorter code than
requiring them to be built from the empty objects using mutation methods. This
is especially the case for lookup tables etc.

\subsection{Function Calls}

Function calls are performed with the standard \code{f(x)} syntax. All functions in
Garter take a fixed number of arguments of fixed types, and produce a result of
a fixed type. Python's default arguments and named arguments were avoided in order to
simplify writing and calling functions for new programmers.

\begin{lstlisting}[language=Python]
def f(x : int): print('hi', x)
f(5) # prints 'hi 5'
\end{lstlisting}

\subsubsection{Magic Functions}

Garter also adapted some functions from Python which require variable count,
variable type, or keyword arguments. These functions were special cased, and
built-in to the language, and are present because of how important they are.

The functions which act this way are:

\begin{itemize}
\item \code{len()} - length of the argument (\code{str}, \code{[T]}, or \code{\{K: V\}})
\item conversion operators (\code{int()}, \code{str()}, \code{float()}) - convert the value to the written type
\item \code{input()} - read in a line from the console, as a string. Optionally takes a prompt strin
\end{itemize}

The \code{print()} function-like statement (Section \ref{sec:print_stmt}),
and \code{range()} function-like form (Section \ref{sec:forstmt}) are also
examples of special-cased functions from Python in Garter.

\subsection{Arithmetic}

Garter supports the standard binary operators (\code{+ - * / // \% **}, mapping
to the operations of addition, subtraction, multiplication, division, flooring
division, modulus, and exponentiation) on both \code{int} and \code{float}.

In addition, the \code{+} binary operator is supported on \code{str}
and \code{[T]}, representing string or list concatenation. It produces a new
value with the value of the concatenation of its arguments.

\code{int} and \code{float} also support the unary \code{+} and \code{-} operators,
which do nothing and negate the value respectively.

This set of operators is fairly standard in programming languages, and is copied
from Python. The decision to seperate flooring division from standard division
(which always produces a \code{float}) is based on Python 3's behavior, and also
helps to prevent errors for new developers which don't understand integer
division.

\subsection{Comparison Operations}

Equality with the equality operators (\code{==}, and \code{!=}) are performed by
value, and supported by all types in Garter. As it is done on value, two lists
or dicts with the same elements, but different identities are still equal.

\begin{lstlisting}[language=Python]
arr1 := [1, 2, 3]
arr2 := [1, 2, 3]
print(arr1 == arr2) # True

dict1 := {'a': 1, 'b': 2}
dict2 := {'a': 1, 'b': 2}
print(dict1 == dict2) # True
\end{lstlisting}

This decision was made as it makes the most intuitive sense. However, there are
exceptions to the comparison-is-by-value rule. Namely, user-defined classes and
functions are compared by reference. 

\begin{lstlisting}[language=Python]
class Foo:
    x := 5
a := Foo()
b := Foo()
print(a == b) # False

def a(): print('hi')
def b(): print('hi')
print(a == b) # False
\end{lstlisting}

The other comparison operators, \code{<}, \code{>}, \code{<=}, and \code{>=}
are supported on the \code{str}, \code{int}, and \code{float} types.

Unlike some other programming languages, comparison operations in Garter are not
binary, but rather n-ary. This means that the operation \code{a < b < c} is not
evaluated as either \code{(a < b) < c} or \code{a < (b < c)}. Instead, it is
evaluated as \code{(a < b) and (b < c)}. This is much closer to the mathematical
meaning of that statement as a constraint, and can help prevent issues when
people with mathematical backgrounds try to write constraints which occur in
other languages. This is important as it helps new programmers which may already
be familiar with mathematical notation to adapt more quickly.

Multiple operators can be mixed together in these comparisons, meaning that
the expression \code{a < b == c < d} is legal in Garter.

\subsection{Attributes and Subscription}

Objects often have attributes, which can be accessed using the \code{.} operator.
This is used to access properties such as methods on built-in and user-defined
types, as well as accessing and potentially modifying attributes on user-defined
types.

\begin{lstlisting}[language=Python]
class Foo:
    x := 5
a := Foo()
a.x = 10
print('a.x =', a.x)
\end{lstlisting}

When accessing values in arrays and dictionaries, subscripting syntax is used.
This involves placing the indexing value in square brackets after the value.
\code{int} is the type used for indexing into \code{[T]}, while \code{K} is the
type used for indexing into \code{\{K: V\}}. Lists in Garter are 0-indexed,
which is the most common indexing strategy in use today, as well as the strategy
used in Python.

\begin{lstlisting}[language=Python]
x := {'a': 5}
y := [5, 10]

x['a'] # 5
y[0] # 5
\end{lstlisting}

Lists may also be sliced into over a range, using the slicing syntax
(\code{[S:E]}). Both the values \code{S} and \code{E} are indexes, and can be
omitted, defaulting to the start and end of the array respectively.

Slices can be assigned to with another array. This will update the values in the
slice to be equivalent to the values from the new array, potentially shifting
around values in the original array.

For more details on this syntax, see \ref{sec:spec_subscription}.

\section{Statements}

\subsection{Functions}
Functions in Garter are defined with the \code{def} keyword. Their syntax is taken
directly from Python, including the type annotation syntax. The Type annotation
syntax is based off of PEP 3107 -- Function Annotations \cite{pythonfuncannot}.
We re-used this syntax as it was already part of the Python language, and thus
was guaranteed not to conflict with other important features, and avoids
adding new syntax as much as possible, even though it is extremely infrequently
used in Python projects.

\begin{lstlisting}[language=Python]
def foo(x : int, y : str) -> bool:
    ...
\end{lstlisting}

Function bodies have a seperate scope, and can be nested. They can reference
variables from their enclosing scope, however, they cannot assign to those
variables without declaring the intention with the \code{global} or
\code{nonlocal} declarations (See \ref{sec:funcdef}).

This helps new developers avoid errors, by requiring them to clarify their
intentions as to whether they want to create a local variable, or mutate an
outer variable, before they use assignment statements.

The \code{return} statement exits a function early, and can be passed a value
when the function is defined to produce a result. The \code{return} statement is
required for functions which provide a result on all code paths.

The variable bindings created by function declarations are immutable, and cannot
be re-assigned to.

\subsection{Control Flow Statements}

\code{for} loops in Garter loop over lists or ranges of numbers. While this is more
restrictive than a traditional range statement, it tends to be more in line with
the standard use of the \code{for} loop. (See \ref{sec:forstmt})

\begin{lstlisting}[language=Python]
for x in range(10):
    # x is 0, 1, 2, 3, 4, 5, 6, 7, 8, 9. Declares x only in scope
    ...

for x in [1, 2, 3]:
    # x is 1, 2, 3. Declares x only in scope
    ...
\end{lstlisting}

Garter also supports the \code{while} loop, which is a basic loop. Every
iteration, including the first iteration, its condition is tested. If it
is \code{False}, then the loop aborts, otherwise it continues
(See \ref{sec:whilestmt}).


The \code{break} and \code{continue} statements can be placed within both
the \code{for} and \code{while} looping construct bodies. \code{break} aborts
the innermost loop, continuing at the next statement after the
loop. \code{continue} causes the next item in the \code{for} loop to be iterated
through, and re-triggers the check in a \code{while} loop.

Garter also supports the \code{if} statement. This statement, like in other
languages, tests a boolean condition. If it is \code{True}, one branch is taken,
otherwise the other branch is taken.

\subsection{Assignments}

In Garter, unlike Python, you cannot assign to a variable which has not been
declared. Assignments are performed with the \code{x = 5} statement. 

Variable declarations are performed with the \code{x := 5} or \code{x : int = 5}
statement. these declarations are lexically scoped, but are not permitted
to shadow other bindings with the same name from the same function.

\begin{lstlisting}[language=Python]
def foo():
    if cond:
        x := 5
    else:
        x := 10
    return x # not OK
    # x is defined within the if's branches, not at the root
\end{lstlisting}

\begin{lstlisting}[language=Python]
def foo():
    x := 0
    if cond:
        x = 5
    else:
        x = 10
    return x # OK
    # x defined in same scope as return, only assigned to in if statements
\end{lstlisting}

\begin{lstlisting}[language=Python]
def foo():
    x := 0
    if cond:
        x := 5 # Not OK
        # This is an error, and catches a problem
        # caused by probably unintentional shadowing
\end{lstlisting}

\begin{lstlisting}[language=Python]
x := 0
def foo():
    x := 10 # OK
    # x is in a different function. Doesn't change global x
\end{lstlisting}

\begin{lstlisting}[language=Python]
x := 0
def foo():
    global x # global statement is required to change x
    x = 10
\end{lstlisting}

Garter also supports closures, allowing functions to access variables defined
in the root of an enclosing function or in the root of the program.

\begin{lstlisting}[language=Python]
def foo() -> int():
    x := 5
    def bar() -> int:
        return x
    return bar
foo()() # 5
\end{lstlisting}

Nested functions are nice for teaching basic functional style programming which
is becoming more popular due to the prevaliance of callback programming with
closures in languages like JavaScript.

Accessing values which aren't in the root of a function, such as the loop
iterator value from a \code{for} loop, is not allowed. This is because the value
may not be valid by the time the function is actually called.

\begin{lstlisting}[language=Python]
def foo(a : [int]) -> [int()]:
    out : [int()] = []
    for x in a:
        def bar() -> int:
            return x # Not OK
            # x is not bound in root of function and thus cannot
            # be seen from enclused function the value x may not
            # have the same value as you are expecting by the time
            # the function is done. The same variable is used
            # for all iterations of the loop, so if this was allowed
            # all functions would return the last value itertated
            # through
        out.push(bar)
    return out
\end{lstlisting}

The declaration syntax (\code{x := a}) is the only syntax addition to Python.
It adds no new semantic meaning to Python, having the same meaning in Python
as a bare assignment statement. The meaning is only useful for the Garter
validation pass.

\subsection{Classes}

Users in Garter are allowed to define their own types, with the \code{class}
construct.

\begin{lstlisting}[language=Python]
class C:
    field := initialvalue

    def method(self):
        # self is implicitly avaliable here as the instance of C!
\end{lstlisting}

When a class is defined, it also declares a function with the name of the class.
This function, when called, will create a new instance of that class.

\begin{lstlisting}[language=Python]
x := C()
\end{lstlisting}

One of the unfortunate design decisions made in Garter because of its Python
legacy is the semantics of field default values. The initial values for fields
are evaluated just once when the class is defined. This can lead to confusing
results when combined with list or dictionary fields, and mutation.

\begin{lstlisting}[language=Python]
class C:
    arr := []
x := C()
x.arr.push(5) # Changes value for default value for arr
y := C()
y.arr # Contains 5
\end{lstlisting}

In Python, the normal way to implement this type of logic would be to define the
magic \code{\_\_init\_\_} method on the class, which is called when a new
instance is created. This would mean that the \code{[]} literal would be
evaluated each time that an instance is created.

\begin{lstlisting}[language=Python]
class C:
    def __init__(self):
        self.arr = []
\end{lstlisting}

This syntax is highly unfortunate, requiring an understanding of constructors
etc. Instead, the syntax of creating type-associated values was chosen, as it
has reasonable semantics, although it sometimes creates undesirable results
when used with mutable values.

There was a consideration early in the development of considering consecutive
\code{:=} statements in classes as being an implicit \code{\_\_init\_\_}
statement. This was chosen to be too different semantically from Python, and
was not chosen as the option.

\subsection{Expressions}

Expressions can also just be written as a statement, and they will be evaluated
for their side effects.

\section{Programs}

A program is just a series of statements. They are executed from top to bottom.
Garter also supports interactive evaluation, in which partial programs are
provided to the validator. Garter programs are validated in a single pass from
top to bottom. Most statements are validate in order. Functions don't have their
bodies typechecked until either:

\begin{enumerate}
\item their binding's name is referenced (for example, to be called), or
\item The end of the current program's input is reached.
\end{enumerate}

This allows mutual recursion, for example:

\begin{lstlisting}[language=Python]
def foo(b: bool):
    if b:
        print('b is true')
    else:
        print('b is false')
        bar() # foo is defined before bar is defined

# if foo was called here, it would be a lookup error, but it isn't called until after so it's OK

def bar():
    foo(True)

foo(False)
\end{lstlisting}

\section{Modules}

Garter only supports a small set of basic modules provided by the base language.
The currently supported modules are \code{random} and \code{turtle}. Expansion
to include support for user- and teacher- defined modules are part of future
work \ref{sec:FutureWork}.

Another important part of teaching would be to provide a simple graphics API
for teaching students, such that they can create simple games, and visualize
progress. Designing such an API is, however, outside the scope of this thesis.


% Chapter 3
\glsresetall % reset the glossary to expand acronyms again
\chapter{Prototype Implementation}\label{ch:Prototype}

In order to validate the claims of simplicity and feasibility, a prototype implementation of the Garter language was devised.

Many options were considered for how to implement this prototype, mostly focusing on compiling the resulting program back down to Python logic, while performing the validation.
See appendix (Alternative Implementation Strategies)

++ Program Structure

Inserted a phase
cpython (lexer) -> cpython (parser) -> garter (validator) -> cpython (compiler) -> cpython (interpereter) (DIAGRAM)

The strategy which was selected was to modify the cpython interpereter, adding the extra syntax required by the Garter programming language to Python's AST (see language design # assignment for info on extra syntax).
Validation pass was written. Takes python's pre-compilation AST and typechecks it. Written in python for ease of development (over C)

New compiler entry point for python programs, like the built in compile function provided by Python itself. Has same functionality, but also performs typechecking

Modify programs such as IDLE (the Python IDE) to instead use this compile function rather than the built-in one. Means that code run in IDLE is now valid Garter code.

Also created command line REPL which uses this entry point - acts like the Python executable.

+++ Why not replace python completely

Cannot completely replace the standard compile pathways with Garter ones - python code is used internally in python and would need to be re-written.

++ REPL Support

REPLs require maintaining scope and type information across compilations. This is done by exposing
an opaque scope object to calling code of compile function. Can be passed to compile function on
a future pass to make the logic run in the same scope as previous scope.

+++ Modifications to REPL mode

Modifications needed to be made to the repl runners such as IDLE in order to ensure that if runtime error killed statement when it is being executed: changes to types in scope are not recorded.

Otherwise, garter validator might think that x has been initialized, when it actually has not yet.



% Chapter 5
\glsresetall % reset the glossary to expand acronyms again
\chapter{Conclusions and Future Work}\label{ch:Conclusion}

\section{Summary of Conclusions}

Garter is a new teaching programming language, based on the Python Programming
Language. It provides the developer with a safe environment in which they can
learn to write programs, while familiarising themselves with Python syntax and
idioms, making the transition into real programming languages used in production
around the world much easier.

\section{Limitations and Future Work}\label{sec:FutureWork}

Garter unfortunately currently has many limitations, mostly due to the short
timeframe in which it was designed and implemented. We hope to shore up these
problems in a future version of Garter.

\begin{enumerate}
\item Garter lacks a module system.

Garter piggybacks on python, but doesn't currently support module loading,
outside of a small pre-defined set of libraries, due to the complexity of
figuring out

\item Garter lacks a mechanism for exposing Python libraries to Garter code

Occasionally a teacher may want to allow their students to perform actions which
are not possible without interacting with other libraries written in Python,
such as interacting with a graphics, windowing, high performance math,
networking or similar library. Garter currently provides no mechanism for
exposing that library other than modifying the distribution directly. Ideally a
mechanism for doing this should be integrated into the language proper.

\item Garter lacks a class inheritance structure

To enter and participate in modern programming, you often need to learn the
mechanics of OOP, specifically single-inheritance with interfaces. No mechanism
was implemented for doing this in Garter (See \ref{sec:Classes}), which means
that teaching of OOP concepts will have to occur in a different language.
A future Garter may perform the design effort to add support for inheritance.

% TODO: REference?

\end{enumerate}




%*******************************************************************************
% BIBLIOGRAPHY
%*******************************************************************************

% Put in \nocite{*} so all entries in the bibliography are included
% \nocite{*}

\phantomsection
\bibliographystyle{plain}
%\bibliography{references/references}
\bibliography{references/proposal}

%*******************************************************************************
% APPENDICES
%*******************************************************************************
\updatechaptername
\appendix

% Chapter 4
\glsresetall % reset the glossary to expand acronyms again
\chapter{Specification}\label{ch:Specification}

\newcommand{\bs}[0]{\textbackslash}

\section{Lexical analysis}

Garter's syntax and lexical analysis are derived from the syntax and lexical rules
from Python. For that reason, many parts of this section are derived directly
from the Python 3 Language Reference \cite{pythonweb}.

Garter programs are described as a series of Unicode code points, and all lexical
analysis is performed on the basis of these Unicode code points.

\subsection{Logical and Physical Lines}

A Garter program is formed of logical lines. The end of a logical line is
represented by the \code{NEWLINE} token. A logical line is formed by one or more
physical line, joined by explicit or implicit joins.

A physical line is a sequence of unicode code points followed by a newline
sequence. This sequence is any of \code{\bs r}, \code{\bs n}, or \code{\bs r\bs n} where \code{\bs r} represents the ASCII Carriage Return (CR) character, and \code{\bs n} represents the ASCII Linefeed (LF) character.

\subsection{Line joining rules}

If the last character in a physical line is the backslash character (\code{\bs}),
and the character is not part of another token (such as a string token), the
current physical line is joined with the next physical line, and no
\code{NEWLINE} token is emitted. Lines which are explicitly joined this way may
not carry comments

If the end of a physical line is reached while inside a pair of parentheses
(\code{()}), square brackets (\code{[]}), or curly braces (\code{\{\}}), the
physical line is implicitly joined with the next line. Lines joined this way may
carry comments.

\subsection{Comments}

A comment starts with the hash character (\code{\#}) which is not part of a
string literal, and ends at the end of a physical line. Comments are ignored by
syntax, and do not cause tokens to be emitted.

\subsection{Indentation}

Leading whitespace at the beginning of a logical line is used to determine the
indentation level of the line, which is used for blocks in the Garter
programming language.

Whitespace measurements are done on spaces, with tabs being replaced by one to
eight spaces, such that the total number of characters is a multiple of 8.

The lexical analysis uses these indentation levels to produce \code{INDENT} and
\code{DEDENT} tokens, using a stack. The following excerpt from the Python
reference explains the algorithm used:

Before the first line of the file is read, a single zero is pushed on the stack;
this will never be popped off again.  The numbers pushed on the stack will
always be strictly increasing from bottom to top.  At the beginning of each
logical line, the line's indentation level is compared to the top of the stack.
If it is equal, nothing happens. If it is larger, it is pushed on the stack, and
one INDENT token is generated.  If it is smaller, it *must* be one of the
numbers occurring on the stack; all numbers on the stack that are larger are
popped off, and for each number popped off a DEDENT token is generated.  At the
end of the file, a DEDENT token is generated for each number remaining on the
stack that is larger than zero.

\subsubsection{Whitespace}

Whitespace between tokens is ignored, unless it contributes to the indentation
rules explained above.

\subsection{Identifiers}
\label{sec:lex_identifiers}

Garter identifiers may be of arbitrary length, and may be formed as follows:

\begin{lstlisting}
identifier ::= start continue*
start      ::= <'a'-'z', 'A'-'Z', '_'>
start      ::= <start, '0'-'9'>
\end{lstlisting}

Implementations may also support additional unicode characters from outside the
ASCII range, following Unicode standard annex UAX-31, like Python.

As an additional restriction, names beginning and ending with two ASCII
underscores (\code{\_\_}), such as \code{\_\_ADD\_\_}, \code{\_\_init\_\_}) are
reserved for implementation use, and may not be written in a Garter program.

\subsubsection{Keywords}

Keywords are a set of identifiers which cannot be used as ordinary identifiers.
Unlike Identifiers, they do not produce a \code{NAME} token, but instead produce
a token specific to the keyword. For any keyword \code{Foo}, the token generated
will be written as \code{'Foo'} in this document.

Some keywords in garter are included for compatibility with the Python
programming language, while others are keywords used only within the Garter
programming language.

The following are the set of keywords in Garter:

\begin{lstlisting}
False      class      finally    is         return
None       continue   for        lambda     try
True       def        from       nonlocal   while
and        del        global     not        with
as         elif       if         or         yield
assert     else       import     pass       print
break      except     in         raise      range
int        float      bool       str        len
\end{lstlisting}

\subsection{Literals}

\subsubsection{String Literals}
\label{sec:string_literals}

String literals takes the following form
\begin{lstlisting}
stringliteral: (shortstring | longstring)
shortstring: ("'" shortstringitem* "'" |
              '"' shortstringitem* '"')
longstring: ("'''" longstringitem* "'''" |
             '"""' longstringitem* '"""')
shortstringitem: shortstringchar | stringescapeseq
longstringitem: longstringchar | stringescapeseq
shortstringchar: <any source character except "\\"
                  or newline or the quote>
longstringchar: <any source character except "\\">
bytesescapeseq: "\\" <any ASCII character>
\end{lstlisting}

Escape sequences in string and bytes literals are interpreted according to rules
similar to those used by Standard C. The recognized escape sequences are:

\begin{center}
  \begin{tabular}{|l|l|}
    \hline
    Escape Sequence & Meaning\\ \hline
    \code{\bs{}newline} & Backslash and newline ignored \\
    \code{\bs\bs} & Backslash (\bs) \\
    \code{\bs{}'} & Single quote (') \\
    \code{\bs{}"} & Double quote (") \\
    \code{\bs{}a} & ASCII Bell (BEL) \\
    \code{\bs{}b} & ASCII Backspace (BS) \\
    \code{\bs{}f} & ASCII Formfeed (FF) \\
    \code{\bs{}n} & ASCII Linefeed (LF) \\
    \code{\bs{}r} & ASCII Carriage Return (CR) \\
    \code{\bs{}t} & ASCII Horizontal Tab (TAB) \\
    \code{\bs{}v} & ASCII Vertical Tab (VT) \\
    \code{\bs{}ooo} & Character with octal value ooo \\
    \code{\bs{}xhh} & Character with hex value hh \\
    \code{\bs{}N\{name\}} & Character named name in the Unicode database\\
    \code{\bs{}uxxxx} & Character with 16-bit hex value xxxx\\
    \code{\bs{}Uxxxxxxxx} & Character with 32-bit hex value xxxxxxxx \\ \hline
  \end{tabular}
\end{center}


All unrecognized escape sequences are left in the string unchanged, including
leaving the backslash in the result.

Multiple consecutive string literals delimited by only whitespace are allowed,
and will be concatenated into a single combined string literal. This allows
writing, for example:

\begin{lstlisting}[language=Python]
x := ("string 1" "string 2")
\end{lstlisting}

Which assigns the value \code{"string 1string 2"} to x.

\subsection{Integer Literals}

The following is the lexical definition for integer literals:

\begin{lstlisting}
integer        ::=  decimalinteger | octinteger |
                    hexinteger | bininteger
decimalinteger ::=  nonzerodigit digit* | "0"+
nonzerodigit   ::=  "1"..."9"
digit          ::=  "0"..."9"
octinteger     ::=  "0" ("o" | "O") octdigit+
hexinteger     ::=  "0" ("x" | "X") hexdigit+
bininteger     ::=  "0" ("b" | "B") bindigit+
octdigit       ::=  "0"..."7"
hexdigit       ::=  digit | "a"..."f" | "A"..."F"
bindigit       ::=  "0" | "1"
\end{lstlisting}

Note that decimal integer literals may not have leading \code{0} characters.
This is to cause c-style octal literals to be an error, as otherwise they would
have unexpected behavior.

\subsection{Floating Point Literals}

Floating point literals expand on integer literals to allow the definition of
non-integral numbers. The following is the lexical definition:

\begin{lstlisting}
floatnumber   ::=  pointfloat | exponentfloat
pointfloat    ::=  [intpart] fraction | intpart "."
exponentfloat ::=  (intpart | pointfloat) exponent
intpart       ::=  digit+
fraction      ::=  "." digit+
exponent      ::=  ("e" | "E") ["+" | "-"] digit+
\end{lstlisting}

\subsection{Operators and Delimiters}

The following are additional special operator and delimiter tokens used by Garter:

\begin{lstlisting}
+       -       *       **      /       //      %
<       >       <=      >=      ==      !=      +=
-=      *=      /=      //=     %=      **=     -> 
,       :       .       ;       =       (       )
[       ]       {       }
\end{lstlisting}

\section{Data Model}

Values in garter are objects. Objects have a type, which defines the type's
fields, and what other operations which may be performed on it. Objects have
identity, and are stored by reference in variables or fields. Assignment doesn't
mutate the object inside of the variable/field, but instead replaces the
reference. A variable or field is immutable if its value cannot be changed.


\subsection{Type Subsumption}
\label{subsumption}

A type \code{T} can be said to subsume another type \code{Q} if a value of type
\code{Q} can be used anywhere that a value of type \code{T} can be used. For
example: an \code{int} is a more specific type than a \code{float}, and it would
be legal to use a \code{int} anywhere that a \code{float} would be required.
Thus we can say \code{float} subsumes \code{int}. In addition, we can say the
inverse, that \code{int} is a subtype of \code{float}.

\subsection{Types}

\begin{lstlisting}
type: 'int' | 'float' | 'str' | 'bool' | NAME |
      '{' type ':' type '}' | '[' type ']' |
      (type | 'None') '(' [typelist] ')'
typelist: type (',' type)* [',']
\end{lstlisting}

\subsubsection{Numbers (\code{int} and \code{float})}

\code{float} is an floating point number. An object of this type may asusme the
value of a double precision floating point value. \code{float} exposes no
attributes. \code{float} is ordered, and can be compared with the comparison
operators \code{<}, \code{>}, \code{<=} and \code{>=}.

\code{int} is an integer. An object of this type may assume the value of an
arbitrary integer value. \code{int} exposes no attributes. \code{int} is a
subtype of \code{float}. \code{int} is ordered, and can be compared with the
comparison operators \code{<}, \code{>}, \code{<=} and \code{>=}.


\subsubsection{Strings (\code{str})}

\code{str} is a unicode string. It may assume the value of an arbitrary length
sequence of unicode codepoints. \code{str} is ordered, and can be compared with
the comparison operators \code{<}, \code{>}, \code{<=} and \code{>=}. \code{str}
exposes the following attributes:

\code{join : str([str])}: The \code{join} attribute is a function object which
concatenates the elements of the argument array, using the implicit self
argument as the seperator. The \code{join} attribute is immutable.

\subsubsection{Booleans (\code{bool})}

\code{bool} is a boolean. It may assume either the value \code{True} or
\code{False}. \code{bool} exposes no attributes.

\subsubsection{Dictionaries (\code{\{K: V\}})}

\code{\{K: V\}} is a one-to-many map from values of type \code{K} to values of
type \code{V}. \code{K} and \code{V} may be arbitrary types. \code{\{K: V\}}
exposes the following attributes. All attributes of \code{\{K: V\}} are
immutable:

\code{pop : V(K, V)}: If the first parameter is a key in the map, Mutates the
map, removing the key from the mapping, and returns the associated value.
Otherwise, returns the second parameter.

\code{setdefault : V(K, V)}: If the first parmaeter is a key in the map, returns
its associated value. Otherwise, mutates the map, inserting tke key-value pair
of the parameters to the map, and returns the second parameter.

\code{get : V(K, V)}: If the first parameter is a key in the map, returns the
associated value. Otherwise, returns the second parameter.

\code{clear : None()}: Mutates the map, removing all key-value pairs.

\code{copy : \{K: V\}()}: Returns a new \code{\{K: V\}}, with a shallow copy
of the key-value pairs from the map.

\code{update : None(\{K: V\})}: Mutates the map, adding all key-value pairs from
the first parameter, overriding any existing conflicting key-value pairs.

\subsubsection{Lists (\code{[T]})}

\code{[T]} is a list of values of type \code{T}. An object of this type may
assume the value of an arbitrary length list of values of type \code{T}.
\code{[T]} exposes the following attributes. All attributes of \code{[T]} are
immutable:

\code{append : None(T)}: Mutates the list, adding the first parameter to the end
of the list.

\code{extend : None([T])}: Mutates the list, adding the elements of the first
parameter to the end of the list.

\code{insert : None(int, T)}: Mutates the list, adding the second parameter
at the index specified by the first parameter, shifting all elements after
the insertion point one element.

\code{remove : None(T)}: Mutates the list, removing the first element for which
\code{it == arg} (where \code{it} is the element, and \code{arg} is the
argument) evaluates to \code{True}.

\code{pop : T(int)}: Mutates the list, removing the element at the index
specified by the first parameter, and returning it. Elements with indexes
greater than the first parameter are shifted down one index.

\code{index : int(T)}: Returns the index of the first element in the list for
which \code{it == arg} (where \code{it} is the element, and \code{arg} is the
argument) evalues to \code{True}.

\code{count : int(T)}: Returns the number of elements in the list for which
\code{it == arg} (where \code{it} is the element, and \code{arg} is the
argument) evaluates to \code{True}.

\code{reverse : None()}: Mutates the list, reversing the order of its elements.

\code{sort : None()}: Mutates the list, sorting it in ascending order. This
attribute is only avaliable on the list types \code{[int]}, \code{[float]}, and
\code{[str]}.

\subsubsection{Functions (\code{T(P, ...)})}

Function objects may have a return type (\code{T}), or if \code{None} is written
instead of a type, the function returns no value. In addition, they have an
arbitrary-length list of parameter types. Function objects do not expose any
attributes.

\subsubsection{User-defined Types}

Users may define their own types using the \code{class} statement (Section
\ref{sec:classdef}). These classes have the attributes as defined by their class
definition. They are unordered, and can be compared for identity with the
equality (\code{==} and \code{!=}) operator.

For more details on user-defined types, see Section \ref{sec:classdef}.

\subsection{Unwritable Types}
\label{writable_types}

A type is said to be an unwritable type if it cannot be written using the type
syntax. Values with unwritable types may not be assigned to variables or used as
the type of fields, parameters, or return values. These types may only be used
as the type of a temporary value within an expression. Unwritable types are the types
of expressions such as \code{None}, \code{[]} and \code{\{\}}.

The expression \code{None} has the unwritable type \code{\_\_NoneType\_\_}. This
type is subsumed by all class types. Thus, the \code{None} value can be used in
place of any class type due to subsumption rules (Section \ref{subsumption}).

The expression \code{[]} has the unwritable type \code{\_\_EmptyListType\_\_}.
This type is subsumed by all list types. Thus, the \code{[]} value can be used
in place of any list type due to subsumption rules (Section \ref{subsumption}).
\code{\_\_EmptyListType\_\_} exposes the same methods as a normal list, except
for those which require knowing the element type.

The expression \code{\{\}} has the unwritable type \code{\_\_EmptyDictType\_\_}.
This type is subsumed by all dict types. Thus, the \code{\{\}} value can be used
in place of any dict type due to subsumption rules (Section \ref{subsumption}).
\code{\_\_EmptyDictType\_\_} exposes the same methods as a normal dict, except
for those which require knowing the key/value types.

All compound list or dictionary types which have keys, values, or item types which
are unwritable are also unwritable. They can be subsumed by any type for which
all literal types match, and the unwritable keys, values, and/or items are subsumed
by the corresponding keys, values, and/or items in the subsuming type.

\section{Program Start Points}

\begin{lstlisting}
single_input: (NEWLINE | simple_stmt |
               compound_stmt NEWLINE |
               defn NEWLINE)
file_input: (NEWLINE | stmt)* ENDMARKER
\end{lstlisting}

A Garter input may either consist of a self-contained program, consisting of a
series of newlines and statements; or a single line of interactive
input coming from, for example, a REPL.

\section{Definitions and Scoping}
\label{sec:defs_and_scoping}

\begin{lstlisting}
defn: funcdef | classdef | vardef
\end{lstlisting}

Definitions may occur wherever a statement is permitted. They bind a value to
the given name in the current scope. Binding is performed on a per-block level.
Shadowing of variables from outside of the current function is permitted, but
shadowing within a function is not allowed. Redeclarations are also prohibited.

When referring to a name, first a check is made to see if the variable is local.
A variable is local if there is a statement declaring that variable within the
current function (whether that variable is currently in scope is ignored). If
the name is nonlocal, enclosing scopes are checked for local variables. If a
variable with the name is not found, it is a validation time error. Once a
variable is found, the variable's validity is checked. A local variable is valid
if it has been declared within the current block or a direct parent. A nonlocal
variable is valid if it is declared above the enclosing function definition in
the root of a function or module. Other references are prohibited as they may no
longer be valid when the function is called.

Variables may not be mutated if they are defined using an immutable declaration
form. Nonlocal variables may not be mutated, although this may be bypassed with
the \code{varfwd} forms described in Section \ref{sec:funcdef}.

\subsection{Function Definition}
\label{sec:funcdef}

\begin{lstlisting}
funcdef: ('def' NAME '(' [paramlist] ')' ['->' type] ':'
          funcbody)
paramlist: param (',' param)* [',']
param: NAME ':' type

funcbody: (simple_stmt | NEWLINE INDENT varfwd* stmt+ DEDENT)
varfwd: globaldef | nonlocaldef
globaldef: 'global' NAME (',' NAME)*
nonlocaldef: 'nonlocal' NAME (',' NAME)*
\end{lstlisting}

A function definition defines a new immutable variable with the given name. It has
the type of the corresponding function object, with the return type specified
after the \code{->} token, or None if no return type is specified. The parameter
types are defined as the types written after the \code{:} token in each parameter.

For example, the function definition:

\begin{lstlisting}
def foo(a: int, b: float, c: str) -> bool:
    ....
\end{lstlisting}

Would define ani immutable variable of type \code{bool(int, float, str)}.

When the function is invoked, the statements in the body of the function are executed
sequentially.

The variable forward forms, \code{globaldef} and \code{nonlocaldef} accept names as
arguments. They state that for name resolution rules, the global, or nonlocal (but in
another function) variables should be treated as though they are local. This means that
they may be mutated, even though they exist in an enclosing nonlocal scope.

Calls to functions are described in more detail in Section \ref{sec:call}.

The body of the function is not validated until either the end of the current
input's validation phase, (either an interactive command or program file), or
the validation of its first reference to the variable within the current input.
This means that functions may refer to other functions or variables which have
not been declared yet, as long as they are declared before the function is used.

\subsection{Class Definition}
\label{sec:classdef}

\begin{lstlisting}
classdef: 'class' NAME ':' class_body
class_body: fielddef | NEWLINE INDENT class_stmt+ DEDENT
class_stmt: fielddef | mthddef

fielddef: NAME ':' [type] '=' expr
mthddef: ('def' NAME '(' mparamlist ')' ['->' type] ':'
          funcbody)
mparamlist: NAME (',' param)* [',']
\end{lstlisting}

A class definition defines a new type to the Garter type system. User-defined
classes may not be referred to in a Garter program before their definition.

When the garter class definition for a type \code{T} is written, an immutable
variable of type \code{T()} is defined with the name \code{T}. This function can
be called to get a new instance of the type \code{T}. In addition, the
initializer statements for the class are run when a class definition is executed
as a statement.

If a class is declared within a function, each time that function is run the
initializer statements for the class will be re-run, and will be used for
new instances of that class within the given scope.

All field and method definitions declare attributes on the class type.

All class types also accept, as a possible value, the value \code{None}. This
represents the absense of a meaningful value. Any attempt to access a field on a
\code{None} value results in a runtime error.

\subsubsection{Field Definitions}
\label{sec:fielddef}
A field definition defines a new mutable attribute with the written type on the
class type. For example, the field definition \code{x : int = 10} would define a
mutable attribute \code{x} on the class type, with type \code{int}. Like with
variable declarations, the type of the field may be inferred if the type is not
written.

\subsubsection{Method Definitions}
\label{sec:methoddef}
A method definition defines a new mutable attribute with a function type on the
class type. The function type will be the same as if the method definition was
written as a function definition (Section \ref{sec:funcdef}), except that the
first parameter need not be typed, and will be implicitly passed a reference to
the instance of the class the function is called on.

Calls are described in more detail in Section \ref{sec:call}.

\subsubsection{Variable Definition}
\label{sec:vardef}

\begin{lstlisting}
vardef: NAME ':' [type] '=' expr
\end{lstlisting}

A variable definition defines a new mutable variable in the current scope. If
the statement is invoked at the global scope, then it defines a new variable in
the global scope. If the statement is invoked within the root of a function, it
defines a new variable inside of that function's scope. The expression is then
evaluated, and assigned to that variable.

If the type is not provided, it is inferred from the type of the expression. This is
occasionally not possible, and a type annotation must be provided (such as with the
\code{None} literal).

\section{Statements}

\begin{lstlisting}
stmt: simple_stmt | compound_stmt | defn
simple_stmt: small_stmt (';' small_stmt)* [';'] NEWLINE
\end{lstlisting}

Control flow in Garter is controlled by statements. These statements can be
categorized into two groups: the single-line small statements, and the
multi-line compound statements.

Definitions are also statements, but are described in Section
\ref{sec:defs_and_scoping}.

\subsection{Small Statements}

\begin{lstlisting}
small_stmt: (expr | assign_stmt | pass_stmt |
             break_stmt | continue_stmt |
             return_stmt | print_stmt)
\end{lstlisting}

\subsubsection{Expr Statement}

The expr statement allows for an expression to be invoked for its side effects,
discarding the resulting value. Usually this is done with a function or method
call. More details on expressions in Section \ref{sec:expr}.

\subsubsection{Assignment Statement}

The basic assignment operator \code{lhs = rhs} assigns the value in \code{rhs}
to \code{lhs}.

Compound assignment operators, \code{+= -= *= /= \%= **= //=} perform their
operation (\code{+ - * / \% ** //} respectively), and the assign the result to
\code{lhs}, evaluating \code{lhs} only once.

The \code{lhs} of the operator must be an assignable target. This is either a
mutable variable, field, array or dictionary index, or array slice. Other
expressions are not legal on the left of an assignment operator. If a field or
variable is immutable (such as is the case with fields corresponding to methods,
and variables corresponding to functions and classes), then it is not a legal
assignable target. If a variable is nonlocal, it is also not a legal assignment
target. To bypass this, the \code{nonlocal} and \code{global} forms may be used
(Section \ref{sec:funcdef}).

\subsubsection{Pass Statement}

\begin{lstlisting}
pass_stmt: 'pass'
\end{lstlisting}

The pass statement does nothing. It exists to enable the definition of empty
blocks.

\subsubsection{Break Statement}
\label{sec:break_stmt}

\begin{lstlisting}
break_stmt: 'break'
\end{lstlisting}

The break statement may only be placed lexically within the for or while
statements. It causes control flow to immediately continue to the next statement
after the for or while statement, not executing the remainder of the current
iteration, and ignoring the normal loop exit conditions.

\subsubsection{Continue Statement}
\label{sec:continue_stmt}

\begin{lstlisting}
continue_stmt: 'continue'
\end{lstlisting}

The continue statement may only be placed lexically within the for or while
statements. It causes control flow to immediately continue to the next iteration
of the loop, not executing the remainder of the current iteration.

\subsubsection{Return Statement}

\begin{lstlisting}
return_stmt: 'return' [expr]
\end{lstlisting}

The return statement may only be placed lexically within a function declaration.
It causes control flow to immediately exit the current function, not executing
the remaining statements in the function. If the return statement is passed an
expression, then it must have the same type as the return type of the function,
and that value is used as the return value of the function. If the return
statement is not passed an expression, then the function must not have a return
type.

\subsubsection{Print Statement}

\begin{lstlisting}
print_stmt: ('print' '(' expr (',' expr)*
             [',' 'end' '=' expr] [','] ')')
\end{lstlisting}

Prints out the result of casting each of the arguments to a \code{str} to the
screen. If this operation would fail, instead prints out a useful debug
representation of the object. By default, each argument's representation is
seperated by an ASCII space character (\code{' '}), and the print statement's
output is terminated with an ASCII newline character (\code{'\bs{}n'}). This
newline character can be replaced by passing the \code{end} 'named argument',
which will instead terminate the string. The \code{end} named argument must be a
\code{str}.

\subsection{Compound Statements}

\begin{lstlisting}
compound_stmt: (if_stmt | while_stmt | for_stmt)
\end{lstlisting}

\subsubsection{If Statement}
\begin{lstlisting}
if_stmt: ('if' expr ':' suite ('elif' expr ':' suite)*
          ['else' ':' suite])
\end{lstlisting}

The if statement is used for conditional execution. It selects exactly one of
the suites to execute by evaluating the expressions one by one until one is
found to evaluate to the boolean value \code{True}. If all expressions evaluate
to \code{False}, then the suite of the else clause, if present, is executed.

All of the expression arguments must be of type \code{bool}.

\subsubsection{While Statement}
\begin{lstlisting}
while_stmt: 'while' expr ':' suite
\end{lstlisting}

The \code{while} statement is used for repeated execution while an expression evaluates
to \code{True}. It will repeatedly test the expression, and if it is
\code{True}, executes the suite. Whenever the expression evaluates to
\code{False}, the loop terminates.

A \code{break} statement (Section \ref{sec:break_stmt}) executed in the suite
terminates the loop immediately. A \code{continue} statement (Section
\ref{sec:continue_stmt}) executed in the suite skips the rest of the suite and
goes back to testing the expression.


\subsubsection{For Statement}
\begin{lstlisting}
for_stmt: ('for' NAME 'in' (expr | range) ':' suite)
range: 'range' '(' expr [',' expr [',' expr]] [','] ')'
\end{lstlisting}

The \code{for} statement is used to iterate over the elements of a list. The
expression is evaluated once, and must have a \code{[T]} type. The suite is then
executed once for each element of the iterator, with the value bound to the
name. This name binding is a local declaration, much like an the assignment
declaration, however it is not valid outside of the body of the for statement
(Section \ref{sec:vardef}.

If the \code{range} form is used instead of the expression, then all of the
expressions must have type \code{int}. In the 1 argument case, the suite is
executed $N$ times, where $N$ is the first argument, with the name bound to each
integer in the range $[0, N)$. In the 2 argument case, the suite is executed
$M-N$ times, where $N$ is the first argument, and $M$ is the second argument,
with the name bound to each integer in the range $[N, M)$. In the 3 argument
case, it acts much like the 2 argument case, except the step is $I$ instead of
1, where $I$ is the 3rd argument.

A \code{break} statement (Section \ref{sec:break_stmt}) executed in the suite
terminates the loop immediately. A \code{continue} statement (Section
\ref{sec:continue_stmt}) executed in the suite skips the rest of the suite and
proceeds to the next item in the list.


\section{Expressions}
\label{sec:expr}

\subsection{Atoms}

\begin{lstlisting}
atom: ('(' expr ')' |
       '[' [exprlist] ']' |
       '{' [dictmaker] '}' |
       'len' '(' expr ')' |
       ('str' | 'int' | 'float') '(' expr ')' |
       NAME | INTEGER | FLOATNUMBER | STRING+ |
       'None' | 'True' | 'False')
exprlist: expr (',' expr)* [',']
dictmaker: expr ':' expr (',' expr ':' expr)* [',']
\end{lstlisting}

Atoms are the building blocks of expressions, and are the base cases for the
recursive definitions of the other expressions.

\subsubsection{Identifiers (names)}
See Section \ref{sec:lex_identifiers} for lexical information on identifiers.
This identifier evaluates to the value of the variable with the given name
in the current scope. The type of this expression is the type of the variable
with the given name.

If there is no variable in the current scope with the given name, it is a
validation-time error.

\subsubsection{Literals}
Literals evaluate to a literal of the given type. The \code{STRING} literal
evaluates to a \code{str} object, The \code{INTEGER} literal evaluates to a
\code{int} object, the \code{FLOAT} literal evaluates to a \code{float}
object, the \code{'True'} literal evaluates to the \code{bool} value
\code{True}, and the \code{'False'} literal evaluates to the \code{bool}
value \code{False}.

The \code{'None'} literal evaluates to the common \code{None} value of
user-defined types. For more information see Section \ref{sec:classdef}.

\subsubsection{Parenthesized Expression}
Parenthesized Expressions evaluate to the value of the contained expression, and
primarially exist as a mechanism for expression precidence and operation orders.

\subsubsection{List literals}

\begin{lstlisting}
list_literal: '[' [exprlist] ']'
exprlist: expr (',' expr)* [',']
\end{lstlisting}

List literals evaluate their contained expressions in order. All expressions
must be of a single type \code{T}. They evaluate to a \code{[T]} value with the
results of the contained expressions as the list elements.

\subsubsection{Dictionary literals}

\begin{lstlisting}
dict_literal: '{' [dictmaker] '}'
dictmaker: expr ':' expr (',' expr ':' expr)* [',']
\end{lstlisting}

Dictionary literals evaluate their contained expressions in order. All
expressions to the left of the colon must be of a single type \code{K}, while
expressions to the right of the colon must be of a single type \code{V}. They
evaluate to a \code{\{K:V\}} value with the results of the expressions to the
left of the ':' as the keys, and the ones to the right as values. Duplicate keys
produce a runtime error.

\subsubsection{Len Expression}

\begin{lstlisting}
len: 'len' '(' expr ')'
\end{lstlisting}

If the expr is a \code{str}, produces the length of the string as an \code{int}.
If the expr is an \code{[T]}, produces the number of items in the list. If the
expr is an \code{\{K: V\}}, produces the number of key-value pairs in the
dictionary. Otherwise, causes a validation time error.

\subsubsection{Typecast Expressions}

\begin{lstlisting}
typecast: ('int' | 'str' | 'float') '(' expr ')'
\end{lstlisting}

The typecast expressions attempt to convert their argument to their type.

All typecast expressions only accept \code{int}, \code{float}, and \code{str}
arguments.

The \code{str} typecast will convert \code{int} and \code{float} objects to
their \code{str} representation. For example, the \code{int} \code{5} will be
converted to the \code{str} \code{"5"}. It will return \code{str} arguments
unchanged.

The \code{int} typecast will truncate \code{float} types to their closest
integer representation, rounded toward \code{0}. It will also attempt to parse
\code{str} types as an \code{int}, causing a runtime error if this fails. It will
return \code{int} arguments unchanged.

The \code{float} typecast will truncate \code{int} types to their closest
floating point representation, which may be less precise than the \code{int}'s
original representation. It will also attempt to parse \code{str} types as an
\code{float}, causing a runtime error if this fails. It will return \code{float}
arguments unchanged.

\subsection{Primaries}
\begin{lstlisting}
primary: fieldref | subscription | call
\end{lstlisting}

\subsubsection{Field Reference}
\label{sec:fieldref}

\begin{lstlisting}
fieldref: primary '.' NAME
\end{lstlisting}

A field reference accesses an attribute of the given object. The type of this
expression is the type of the attribute on the object. If the object lacks
an attribute with the given name, it is a validation-time error.

\subsubsection{Call}
\label{sec:call}
\begin{lstlisting}
call: primary '(' [exprlist] ')'
\end{lstlisting}

A call calls the function object with the passed arguments. The primary must
evaluate to a function object, and the passed arguments must have types which
correspond to the argument types of the function object.

The result type is the return type of the function object. If the function
object has no return type, then this expression has no result.

This is also used to call methods, as methods are defined as immutable fields
with a function object type.

\subsubsection{Subscription}
\begin{lstlisting}
subscription: primary '[' (expr | [expr] ':' [expr]) ']'
\end{lstlisting}

If the first form of subscription is used, the primary must evaluate to a
\code{[T]} or \code{\{K:V\}}. If the primary is a \code{[T]}, then the expr must
be a \code{int}, and the expression will yield the nth element of the list,
where n is the value of the expr. If the primary is a \code{\{K:V\}}, then the
expr must be a \code{K}, and the expression will yield the \code{V} which is
associated with the given key.

If the second form of subscription is used, then the primary must evaluate to a
\code{[T]}, and both expressions, if provided, must be a \code{int}. The first
argument defaults to \code{0}, and the second defaults to the value of
\code{len(primary)}. The expression yields a new list, containing the values in
the list starting at the index of the first argument, and ending before the
element at the index of the second argument.

Negative integer indexes on \code{[T]} are offset from the end of the list, so
\code{-1} is the index of the last element in the list.

\subsection{Comparison Operators}
\begin{lstlisting}
comparison: arith (comp_op arith)*
comp_op: '<'|'>'|'=='|'>='|'<='|'!='|'in'|'not' 'in'
\end{lstlisting}

These operators are not parsed as left-associative, unlike the Arithmetic
Operators below. Instead, a sequence of comparison operators such as \code{a < b
  == c > d} will be resolved like \code{(a < b) and (b == c) and (c > d)},
except that each expression will only be evaluated once.

These operators all produce an expression of type \code{bool}, and will be
written with only two operands. Their semantic meaning in sequence is as is
written above. Their valid types will be written in place of their operands. As
\code{int} is subsumed by \code{float}, it may be used whenever a \code{float}
is required.

\code{float < float}: \code{True} if $lhs < rhs$, \code{False} otherwise.

\code{float > float}: \code{True} if $lhs > rhs$, \code{False} otherwise.

\code{float <= float}: \code{True} if $lhs \le rhs$, \code{False} otherwise.

\code{float >= float}: \code{True} if $lhs \ge rhs$, \code{False} otherwise.

\code{float == float}: \code{True} if $lhs = rhs$, \code{False} otherwise.

\code{str == str}: \code{True} if $lhs$ and $rhs$ represent the same unicode
sequence, \code{False} otherwise.

\code{bool == bool}: \code{True} if $lhs$ and $rhs$ have the same value,
\code{False} otherwise.

\code{[T] == [T]}: \code{True} if $lhs$ and $rhs$ have the same number of
elements, and for all elements $i$, \code{lhs[i] == rhs[i]}.

\code{\{K: V\} == \{K: V\}}: \code{True} if $lhs$ and $rhs$ have the same set of
keys, and for each key $k$, \code{lhs[k] == rhs[k]}.

\code{T == T}: \code{True} if $lhs$ and $rhs$ have the same identity.

\code{T != T}: \code{True} if \code{T == T} is \code{False}, \code{False}
otherwise.

\code{T in [T]}: \code{True} if there is an element $i$ in $rhs$ such that
\code{lhs == rhs[i]}. \code{False} otherwise.

\code{K in \{K: V\}}: \code{True} if the key $lhs$ is in $rhs$, \code{False}
otherwise.

\code{T not in [T]}: \code{False} if there is an element $i$ in $rhs$ such that
\code{lhs == rhs[i]}. \code{True} otherwise.

\code{K not in \{K: V\}}: \code{False} if the key $lhs$ is in $rhs$, \code{True}
otherwise.

\subsection{Arithmetic Operators}
\begin{lstlisting}
arith: mult (('+'|'-') mult)*
mult: unary (('*'|'/'|'%'|'//') unary)*
unary: ('+'|'-') unary | power
power: primary ['**' unary]
\end{lstlisting}

These operators will be written with their valid types in place of their
operands. Any unlisted type-operand combinations are validation-time errors.
With any operators which accept \code{float} also accept \code{int} in place of
their operands, unless a more specialized operator is avaliable. For example
the \code{-} binary operator accepts any combination of \code{int} and
\code{float} operators, and produces an \code{int} if both operands are
\code{int}, and a \code{float} otherwise, but only has two entries on this list,
one for \code{int - int}, and one for \code{float - float}.

\code{int + int}: Returns the value $lhs + rhs$ as an \code{int}.

\code{float + float}: Returns the value $lhs + rhs$ as a \code{float}.

\code{str + str}: Returns a \code{str} containing the concatenation of the two strings.

\code{[T] + [T]}: Returns a new \code{[T]} containing the concatenation of the two lists.

\code{int - int}: Returns the value $lhs - rhs$ as an \code{int}.

\code{float - float}: Returns the value $lhs - rhs$ as a \code{float}.

\code{int * int}: Returns the value $lhs \times rhs$ as an \code{int}.

\code{float * float}: Returns the value $lhs \times rhs$ as a \code{float}.

\code{float / float}: Returns the value $\frac{lhs}{rhs}$ as a \code{float}.

\code{int \% int}: Returns the remainder from the division $\frac{lhs}{rhs}$ as
an \code{int}. The sign of this remainder will match the sign of the second
operand.

\code{float \% float}: Returns the remainder from the division $\frac{lhs}{rhs}$
as an \code{float}. The sign of this remainder will match the sign of the second
operand.

\code{int // int}: Returns the value $\frac{lhs}{rhs}$, rounded down to the
nearest integer as an \code{int}.

\code{float // float}: Returns the value $\frac{lhs}{rhs}$, rounded down to the
nearest integer as an \code{float}.

\code{float ** float}: Returns the value $lhs^{rhs}$, as a \code{float}.

\code{+ int}: Returns its operand unchanged.

\code{+ float}: Returns its operand unchanged.

\code{- int}: Returns the negation of its operand.

\code{- float}: Returns the negation of its operand.

\subsubsection{Unary Operators}

\code{+}: If the operand is \code{int} or \code{float}, returns it unmodified.
Other operand types cause a validation time error.

\code{-}: If the operand is \code{int} or \code{float}, returns its negation with
the same type. Other operand types cause a validation time error.

\subsection{Boolean Operators}

\begin{lstlisting}
or_expr: and_expr ('or' and_expr)*
and_expr: not_expr ('and' not_expr)*
not_expr: 'not' not_expr | comparison
\end{lstlisting}

Boolean Operators, unlike the other operators, do not necessarially evaluate
all of their operands, they may short-circuit their execution if the value of
their left operand is sufficient to compute the result.

All Boolean operators require all operands to have type \code{bool}.

The \code{or} operator evaluates its left operand. If it produces the boolean
value \code{True}, the result is \code{True}. Otherwise, it evaluates its
right operand, and produces its result.

The \code{and} operator evaluates its left operand. If it produces the boolean
value \code{False}, the result is \code{False}. Otherwise, it evaluates its
right operand, and produces its result.

The \code{not} operator evaluates its operand. If it produces the boolean value
\code{False}, the result is \code{True}. Otherwise, the result is \code{False}.

\subsection{If Expression}
\begin{lstlisting}
expr: or_expr ['if' or_expr 'else' expr]
\end{lstlisting}

The first and third operands must have a common type \code{T}. The second operand
must have type \code{bool}.

The \code{if} expression evaluates its second operand. If it produces the
boolean value \code{True}, the first operand is evaluated, and its result is
produced. Otherwise, the third operand is evaluated, and its result is produced.

\section{Builtin Functions}
\label{sec:functions}

Garter defines only one function in the global scope at the start of the program:

\code{input : str()}: Reads in a line of textual input from the user, producing
a \code{str} containing the read-in text.

\code{ord : int(str)}: Returns the ordinal number for the passed-in character.


% Appendix 1
\glsresetall % reset the glossary to expand acronyms again
\chapter{Grammar}\label{ch:Grammar}

\lstinputlisting[language=Python]{Grammar}

% Appendix 3
\glsresetall % reset the glossary to expand acronyms again
\chapter{Error Glossary}\label{ch:ErrorGlossary}

The error messages are an important part of Garter. Garter programs will be
written by inexperienced programmers, and the error messages which occur when
they write incorrect programs will have to be useful and provide a mechanism for
correcting their errors.

Sometimes, an error will be connected with 'notes', which are intended to
provide additional information for the coder, directing them towards advice. For
this reason, the error messages will occasionally be long, spanning multiple
lines. To avoid scaring away developers with pages of errors, only one error
should be shown at a time.

Only validation-time errors are described here. 

\section{Operator Type Mismatch}
\begin{lstlisting}[breaklines]
Line X, Column X
    ....
     ^
OperatorTypeMismatch:
The operands to operator '+' (int + str), are invalid.
\end{lstlisting}

This error appears when a user attempts to use a unary or binary operator on two types
which are not supported.

If the operator is '+', and at least one of the operands is \code{str}, then we
can predict that the user meant to perform string concatenation, and provide
a fixit note.

\begin{lstlisting}[breaklines]
NOTE: Can only concatenate str and str. Consider casting the left argument
to a str for concatenation with 'str(...)'. 
\end{lstlisting}

\section{Parameter Type Mismatch}
\begin{lstlisting}[breaklines]
Line X, Column X
    ....
     ^
ParameterTypeMismatch:
The 3rd parameter to this function is invalid. Expected int, instead found bool.

NOTE: Function expects parameters (int, int, float)
\end{lstlisting}

This error appears when a user attempts to call a function, either internal or user-defined,
with the incorrect type parameter.

The expected parameters are listed to help the user correct their logic quickly.

\section{Parameter Count Mismatch}
\begin{lstlisting}[breaklines]
Line X, Column X
    ....
     ^
ParameterCountMismatch:
This function expects 3 parameters, instead found 4 parameters.

NOTE: Function expects parameters (int, int, float)
\end{lstlisting}

This error appears when a user attempts to call a function, either internal or user-defined,
with the incorrect number of arguments.

The expected parameters are listed to help the user correct their logic quickly.

\section{No Such Attribute}
\begin{lstlisting}[breaklines]
Line X, Column X
    ....
     ^
NoSuchAttribute:
This object of type 'str' has no attribute named 'jion'.

NOTE: 'str' has attributes: 'join'.
\end{lstlisting}

This error appears when a user attempts to access an attribute of an object
which is not available. The list of available attributes is listed to help them
find the attribute they ment to access.

\section{Attribute Already Defined}
\begin{lstlisting}[breaklines]
Line X, Column X
    ....
     ^
AttributeAlreadyDefined:
This class has a duplicate definition of the attribute 'foo'.

NOTE: The previous definition of this attribute is here:
Line Y, Column Y
    ....
     ^
\end{lstlisting}

This error occurs when a user attempts to define an attribute on an object which
has previously been defined. It identifies the location of the previous
definition, so that the user can quickly correct the offender.


\section{Variable Already Defined}
\begin{lstlisting}[breaklines]
Line X, Column X
    ....
     ^
VariableAlreadyDefined:
The variable 'foo' has already been defined in this scope.

NOTE: The previous definition of this variable is here:
Line Y, Column Y
    ....
     ^
\end{lstlisting}

This error occurs when a user attempts to define an variable in a scope where it
has previously been defined. It identifies the location of the previous
definition, so that the user can quickly correct the offender.

\section{Invalid Variable}
\begin{lstlisting}[breaklines]
Line X, Column X
    ....
     ^
InvalidVariable:
The variable 'foo' may not have a valid value when it is accessed here.
\end{lstlisting}

This error occurs when a user attempts to access a variable which may not be
valid when it is accessed. This might happen if, for example, a function is
declared in a loop, and the function attempts to access the iterator value.

\section{Incomplete Type}
\begin{lstlisting}[breaklines]
Line X, Column X
    ....
     ^
IncompleteType:
Unable to infer the type of this value. Try annotating this declaration with the desired type.
\end{lstlisting}

This error occurs when the user attempts to assign an unwritable type to a variable or field.
For example, the declaration \code{x := []}, for which the item type of the expression cannot be determined.

If it is a simple error like \code{x := []}, \code{x := \{\}}, or \code{x := None} then we also add one of these notes:
\begin{lstlisting}[breaklines]
NOTE: Annotate this declaration to specify the item type of the list
NOTE: Annotate this declaration to specify the key and value types of the dict
NOTE: Annotate this declaration to specify the class type
\end{lstlisting}

\section{Invalid Typecast Source}
\begin{lstlisting}[breaklines]
Line X, Column X
    ....
     ^
InvalidTypecastSource:
Cannot cast to type 'float' from type 'bool'.
\end{lstlisting}

This error occurs when a user attempts to typecast from an invalid type. It identifies
both of the types.


\section{Mismatched Branch Types}
\begin{lstlisting}[breaklines]
Line X, Column X
    ....
     ^
MismatchedBranchTypes:
The types of the branches of the if expression must match.
Instead found 'int' and 'bool'
\end{lstlisting}

This error occurs when an if expression with mismatched types on the then and
else branches is written.

\section{Not In Loop}
\begin{lstlisting}[breaklines]
Line X, Column X
    ....
     ^
NotInLoop:
The 'break' and 'continue' statements may only be written in loops.
\end{lstlisting}

This error occurs when the user writes a \code{break} or \code{continue} statement
outside of a loop.

\section{Return Outside Function}
\begin{lstlisting}[breaklines]
Line X, Column X
    ....
     ^
ReturnOutsideFunction:
The 'return' statement may only be written within functions.
\end{lstlisting}

This error occurs when the user writes a \code{return} statement outside of a function or method.

\section{Invalid Len Argument}
\begin{lstlisting}[breaklines]
Line X, Column X
    ....
     ^
InvalidLenArgument:
The len expression only accepts lists, dicts, and strs. Instead found a 'bool'.
\end{lstlisting}

This error occurs when an invalid argument is passed to the len magic function.

\section{Invalid Return Type}
\begin{lstlisting}[breaklines]
Line X, Column X
    ....
     ^
InvalidReturnType:
This function must return type 'bool', instead found a 'str'.
\end{lstlisting}

This error occurs when a returns statement in a function has an invalid type.

If the function shouldn't return any type, the error will instead be:

\begin{lstlisting}[breaklines]
This function doesn't return a type, instead found a 'str'.
\end{lstlisting}

If the function should return a type but there is not one written

\begin{lstlisting}[breaklines]
This function must return type 'bool', instead found an argument-less return statement
\end{lstlisting}

\section{Invalid Conditional}
\begin{lstlisting}[breaklines]
Line X, Column X
    ....
     ^
InvalidConditional:
The type of the conditional in an if or while must be 'bool'. Instead found 'int'
\end{lstlisting}

This error occurs when a non-\code{bool} value is used as the conditional in an
\code{if} or \code{while}.

\section{Mismatched List Type}
\begin{lstlisting}[breaklines]
Line X, Column X
    ....
     ^
MismatchedListType:
The type of elements in a list must be consisitent.
The first element in this list literal is 'bool', while the 3rd element is 'str'
\end{lstlisting}

This error occurs when a list literal is used with inconsistent item types.

\section{Mismatched Dict Type}
\begin{lstlisting}[breaklines]
Line X, Column X
    ....
     ^
MismatchedDictType:
The type of keys and values in a dict must be consisitent.
The first key in this dict literal is 'bool', while the 3rd key is 'str'
\end{lstlisting}

This error occurs when a list literal is used with inconsistent key/value types.

\section{Invalid Print Line End}
\begin{lstlisting}[breaklines]
Line X, Column X
    ....
     ^
InvalidPrintLineEnd:
The 'end' named parameter to the print statement must be 'str', instead found 'bool'
\end{lstlisting}

This error occurs when the \code{end} named parameter to the \code{print} function is
the incorrect type.

\section{Invalid Index Type}
\begin{lstlisting}[breaklines]
Line X, Column X
    ....
     ^
InvalidIndexType:
Only 'int' may be used to index into '[bool]'. Instead found 'float'
\end{lstlisting}

This error occurs when the incorrect type is used for indexing into \code{list},
\code{dict} or \code{str} types with the subscription syntax.

\section{Unsupported Index}
\begin{lstlisting}[breaklines]
Line X, Column X
    ....
     ^
UnsupportedIndex:
Cannot index into type 'bool'.
\end{lstlisting}

This error occurs when the indexing form of the subscription syntax is used on a
type which doesn't support it.

\section{Unsupported Slice}
\begin{lstlisting}[breaklines]
Line X, Column X
    ....
     ^
UnsupportedSlice:
Cannot slice into type 'bool'.
\end{lstlisting}

This error occurs when the slicing form of the subscription syntax is used on a
type which doesn't support it.

\section{Invalid Assign Target}
\begin{lstlisting}[breaklines]
Line X, Column X
    ....
     ^
InvalidAssignTarget:
This expression is immutable, and thus cannot be assigned to.
\end{lstlisting}

This error occurs when the user attempts to assign to an expression which isn't
a valid assignment target.

% Appendix 4
\glsresetall % reset the glossary to expand acronyms again
\chapter{Example Programs}\label{ch:ExamplePrograms}

The following are a series of example programs written in the Garter programming
language. The majority of these example programs were adapted from the series of
example programs to be shown to teachers interested in using the Turing
programming for education.

\section{Hello World}

This program demonstrates a minimal program in Garter. It shows the lack of
boilerplate in basic Garter programs, and demonstrates the basic use of
the \code{print} statement.

\lstinputlisting[language=Python]{figs/helloworld.py}

\section{Fibonacci}

This program demonstrates function definitions in Garter, recursion support
and the if statement.

\lstinputlisting[language=Python]{figs/fibonacci.py}

\section{Adder}

This program demonstrates user input through the \code{input} function-like
expression, variable declaration, type coersions, and the \code{while} loop.

\lstinputlisting[language=Python]{figs/adder.py}

\section{Sorter}

This program demonstrates user input through the \code{input} function-like
expression, lists, typed varaible declarations, and the \code{for} loop.

\lstinputlisting[language=Python]{figs/sorter.py}

\section{Linked List}

This program demonstrates a basic implementation of a linked-list structure,
along with some basic implementation methods. It demonstrates class
declarations, and method declarations.

\lstinputlisting[language=Python]{figs/linkedlist.py}

\section{Boxes}

This program demonstrates basic input, string manipulations, and the use of the
\code{range} looping construct.

\lstinputlisting[language=Python]{figs/boxes.py}

\section{Expression Parser}

This program demonstrates the implementation of a trivial expression parser.

\lstinputlisting[language=Python]{figs/expn.py}

\section{Hanoi}

This program demonstrates an implementation of the tower of hanoi algorithm.

\lstinputlisting[language=Python]{figs/hanoi.py}

\section{Random Number Generation}

This program demonstrates the generation of a random numbers using Garter.

\lstinputlisting[language=Python]{figs/rand.py}


% Appendix 2
\glsresetall % reset the glossary to expand acronyms again
\chapter{Alternative Implementations}\label{ch:AltImpls}

The first mechanism for implementing this system would be to follow in the
footsteps of mypy. Mypy uses Python 3's Function Annotation
syntax \cite{pythonfuncannot} to allow code written for mypy to also be valid
python 3 code. On one hand, this is an advantage, as it means that no
translation work needs to be done. Unfortunately, to make typing more explicit
and reduce programmer error, we also want to add annotations to variable
declarations, and other places which are not covered by the function annotation
syntax. However, we will likely re-use the function annotation syntax for this
thesis as a standard way to write function types and return values, as it
already fits with the rest of python's syntax, and is an advanced enough feature
that it is reasonable to remove from the dialect created for this thesis.

Languages based very closely on the semantics of other languages, such as
TypeScript mentioned above, are often implemented through source-to-source
compilers, or transpilers. The transpiler acts like a compiler, parsing and
analyzing the code written in the dialect. However, instead of performing
translation to assembly or machine code directly, transpilers emit code in the
base language. For example, the TypeScript compiler reads in TypeScript code,
analyzes it, and then emits valid ECMAScript code. This has an awesome advantage
of eliminating the requirement of implementing a full language implementation of
the base language, as the semantics can be derived from an existing
implementation of that language. We plan to use that technique for this thesis
in order to avoid re-implementing much of python, and to ensure that we maintain
accurate semantics.

One approach to the development of this dialect would be to take the same
approach as Microsoft. The dialect’s transpiler would be written, much like any
other compiler, from scratch using standard compiler technologies. The code
generation stage would be the only difference between the transpiler and any
other compiler, in that instead of generating machine code, it would be aiming
to generate code in the base language which as closely maps to the original code
as possible, such that errors generated by the interpreter map back accurately
to the source code for debugging purposes. However, writing a complete compiler
from scratch is often unnecessary as people have developed awesome tooling for
implementing language dialects already.

An interesting and useful tool for defining new languages without writing a
complete compiler is the Spoofax
workbench \cite{spoofaxpaper}, \cite{spoofaxweb}. Spoofax is a complete
tool-chain for defining the syntax and semantics of a language. Syntax is
defined using the Syntax Definition Formalism (SDF3) specification format, which
allows for complete language grammars to be described using a declarative
formation. Semantic analysis is handled using a variety of tools, including Name
Binding Language (NaBL), which handles semantic analysis of name binding, such
as namespaces, declarations, and references; and Type Specification Language
(TS), which allows for a declarative definition of a language's type system.
Finally, language developers translate the complete, type-checked and analyzed
language into a base language, such as Java, using the Stratego Transformation
Language \cite{strategoweb}, which declaratively maps AST structures in the new
language to structures in an existing language, such as Java. Spoofax also
provides a comprehensive testing story, with the Spoofax Testing Language (SPT)
which can be used to test parsing rules, error and warnings in language
productions, and transformation outputs. Finally, Stratego also uses these
analyses to generate IDE tooling for the languages in question - languages built
with Stratego get syntax highlighting, go to definition, and other useful tools
almost for free.

The biggest, and most impressive, advantage of the Spoofax toolbox for language
definition is it’s polish and integration with the Eclipse IDE. Languages
defined in spoofax are developed in a complete development environment, which
includes features such as inspection and debugging of intermediate data
structures, as well as IDE integration for the generated language. The IDE
integration is especially useful for a learning project, as it gives new
developers the tools they need to explore and learn about their new programming
language without having to memorize function and type names from libraries.

The Turing Extender Language (TXL) \cite{txlpaper}, \cite{txlweb} is another
system for defining new languages in terms of base languages. TXL’s processing
is split into two distinct steps. First, in the syntactic phase, the dialect
language is parsed, and an AST of the dialect language is generated. This
dialect AST is then passed to the semantic phase, which describes the semantic
meaning of the dialect language through a series of transformations into the
base language AST. This AST is then written out, and can be compiled or
executed.

Much like Spoofax, TXL is intended to be used with known syntax definitions for
a base language, such as the Turing programming language. TXL supports tooling
for modifying portions of a base language’s syntax, allowing for dialect
languages to be easily extended off of the syntactic structures of the base
language. However, TXL is also capable of describing complete new languages,
with completely different syntactic properties. In addition, by using multiple
passes which annotate expressions with their type, it is possible for TXL to
perform complex semantic type analysis. In the case of a dynamic language, a TXL
definition of the base language, such as python, would be developed, followed by
the addition of the typing syntactic structures in the dialect.

Much like Spoofax, and despite the use of Turing in its name, TXL is a generic
framework for extending languages, and is not specific to the Turing Programming
Language. With TXL it is possible to manipulate and generate code in many
different programming languages.

Unlike Spoofax, TXL doesn’t provide an immediate tool for integration into an
IDE, but with a language like Python, an extension to the open-source IDE IDLE
to transform the language before running it shouldn’t be too difficult to add,
and could create a very integrated experience which will be familiar when the
student moves away from the statically typed subset to taking advantage of the
entire language.

The Rascal meta-programming Language \cite{rascalpaper}, \cite{rascalweb} is an
attempt to develop a standard language for all forms of program analysis,
parsing, and code generation. It provides a single set of abstractions for
creating static analyses for existing programming languages, performing
automated code refactorings with understanding of the semantic meaning of the
underlying code, and generating DSLs (Domain Specific Languages) which compile
to an underlying language. Like Spoofax, the programs written in Rascal can also
be connected to an IDE, to provide auto-completion and features like
go-to-definition to developers in the IDE, which is a very useful feature for
developing a teaching language. The unified structure provides an advantage for
the development of

In addition Rascal comes with parsers for existing languages, such as Java,
which can allow for the easy creation of extra tooling or language features
which build on top of existing language parsers, limiting the amount of
development which is necessary in order to add a new feature to a programming
language. This means that new features can be added to a language like Java with
relative ease, by defining the new language feature in terms of features in the
base language. While there are existing libraries for parsing many languages,
Python is supported, which means that a new library will have to be implemented
in Rascal for this project if Python is chosen as the target language, and
Rascal as the transpilation framework.

If we’re OK with building the analysis upon other languages, we might want to
take a look at the Scheme programming language \cite{schemebook}. The Scheme
programming language takes a different approach to language extension then the
approaches taken by Spoofax and TXL: namely, instead of starting from a complex
language syntax and extending it, it starts with an extremely simple syntax -
that of S-expressions, and defines the entire language in terms of them. In
scheme, the statement `if` looks like `(if THEN ELSE)` which is identical to the
appearance of a function call with two arguments `(func ARG1 ARG2)`. This
uniform syntax allows for very easy to use macros. Scheme defines macros like a
special function type which is evaluated outside-in at “compile-time”. When the
macro is invoked, the AST of the arguments is provided to the macro’s
implementation, which is then given the opportunity to emit arbitrary code as
output. In addition, Scheme comes with a powerful macro-by-example system called
define-syntax, which allows for syntactic patterns to be defined and matched
against, along with simple syntax for defining the resulting meaning in a
declarative manner. In addition, scheme macros implement “hygiene”, which means
that they can internally use identifiers without the risk of the identifiers
conflicting with identifiers found at the use-sites of the
macro \cite{schemesyntaxpaper}.

This is a very powerful approach, as it allows defining brand new language
constructs which look nearly identical to the built-in language constructs
directly within the language itself. Unfortunately, it is a relatively poor tool
for defining brand new languages. While it is possible to define many new
language concepts, the syntax is constrained to use s-expressions, and certain
keywords are defined by the core language implementation, and cannot be
re-defined by macros. Thus, scheme’s macro system is a very powerful system for
language extension, but is incapable of defining arbitrary new languages.

However, we are not aiming to create an arbitrary new language, we are looking
to extend the language with type checking. Unfortunately Scheme doesn’t come
with the best tools for the job here, as the program as a unit is not implicitly
wrapped in a macro, which means that it is not possible in the generic case to
perform full-program transformations. However, if an implicit outer macro
invocation was created, it would be possible to define a macro transformation
which would generate valid Scheme, and which would perform type checking.
Unfortunately, however, much of the benefits of the modular scheme approach
would not be able to be taken, and the macro would effectively amount to a
partial compiler implementation.

In addition, Scheme has the disadvantage of, while it has a somewhat significant
library collection, lacking the industry presence which we are looking for to
make the language more appealing to students, and a better stepping stone to
real projects. These factors make Scheme a poor choice for this project.

The Julia Programming Language \cite{juliapaper}, \cite{juliaweb}, much like in
the Scheme programming language, allows programmers to manipulate the AST of
Julia programs. However, unlike Scheme, Julia doesn’t limit its program syntax
to only take the form of s-expressions, rather the syntax takes the form of
arbitrary Julia syntactic structures. The syntax for Julia’s macro expressions,
however, are limited. Macro names must be prefixed by the @ sign, distinguishing
them from native language constructs. In addition they must take one of two
forms, either `@name a1 a2 …` or `@name(a1, a2, …)`. Much like Scheme’s macros,
Julia’s macros are hygienic, and do not pollute the call site’s identifier
space. The arguments which are passed to the the Julia function take the form of
julia expression AST nodes.

In addition to Julia’s macros, Julia also supports generated functions.
Generated functions are formed using the @generated macro, which annotates a
function declaration. Wherever a generated function is called, the generated
function’s body is called, and passed the computed types of the arguments. The
function is then able to generate custom logic which will be executed when the
function is called at run-time. This is very similar in concept to generic
functions in other languages such as C++, except instead of writing the code
generation logic in a declarative system such as C++’s template syntax,
generated function code generation logic is instead written directly in Julia.
This is extremely useful for writing complex generic functions which need to
have custom functionality based on types which would be difficult or impossible
to express under a more limited system. Combined with Julia’s macros, this
provides Julia with a very powerful meta-programming story. However, as is the
case with Scheme, Julia is limited in that it is not possible to define a
completely distinct language with only these constructs, as the syntactic
transformations which may be performed are limited.

Julia is also a type-safe language, and has remarkably good error messages. Many
of the type-checking semantics and error messages which Julia provides could be
used as guides for the target behavior of whatever mechanism is chosen.
Unfortunately, we do not believe that Julia does a good job of filling this
role, as it lacks the libraries and production usage of other languages like
Python due to its young age. However, we could see Julia potentially being used
in the future for teaching programming concepts.

Another area of CS research related to program transformation is the area of
Aspect Oriented Programming. In the Aspect Oriented Programming paradigm,
programs are described in two parts: the main program, which describes the
primary business logic, and a set of “cross-cutting concerns” or
aspects \cite{aspectpaper}. The aspect interpreter or compiler then executes the
main program, interleaving the aspect logic whenever certain “pointcuts”
(language structures) are reached in execution. For example, an aspect may bind
to method calls on a database object, causing a logging operation to take place
whenever one of these calls is made. This technology can also be used by
language developers which wish to make changes to the basic semantics of the
target language. By binding and executing different instructions before and
after pointcuts in the program using an aspect oriented interleaver, the
semantics of individual instructions can be modified without needing to change
the language’s structure, allowing new languages to be defined with different
semantics, but the same syntax, as other languages.

This form of transformation, however, is not the type of language transformation
that we need for this project. Our goal is to avoid changing the semantics of
the language while providing additional analysis at compile time, which takes
advantage of new syntax. For this reason, the technologies used by aspect
oriented programming will be unlikely to be very useful in this project.



%*******************************************************************************
% INDEX
%*******************************************************************************
% Here's where the index would be printed, if you created one.  Default: no.
% Remove the % on the next line to enable.
%\printindex


%********************t***********************************************************
% End DOCUMENT
%*******************************************************************************
\end{document}
