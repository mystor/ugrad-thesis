% Chapter 1
\glsresetall % reset the glossary to expand acronyms again
\chapter{Introduction}\label{ch:Introduction}

% Motivation
\section{Motivation}

Modern software development has moved toward the increased usage of dynamic
programming languages. These languages do not contain a complete semantic
analysis phase during compilation, and instead handle and report any type errors
at run-time, as incorrect code is executed. This serves as an advantage for
programmers due to the ability to easily write generic code (as all code is, by
default, generic over all compatible types), as well as not have to design a
complete data layout before beginning to write code. This allows for projects to
be written quickly and easily. In program contexts where the optimizations only
available to statically typed languages are not necessary, these can be
significant benefits.

Unfortunately, this dynamic execution doesn't only have run-time costs. It also
affects the ease of understanding program errors, usually quite negatively. As
an example, the same basic programming errors have been introduced into two
programs: we will compare the error outputs of two similar incorrect problems,
one written in Python, which is a dynamic language, and the other written in
C \cite{ansic}, which is a static language. The python programs are executed
with the cpython 2.7.10 \cite{cpython} interpreter, and the C programs are
compiled using clang \cite{clangweb}, with the \code{-Werror} flag enabled.

\begin{minipage}[t]{0.5\textwidth}
\lstinputlisting[language=Python]{figs/python1.py}
\end{minipage}
\begin{minipage}[t]{0.5\textwidth}
\lstinputlisting[language=C]{figs/c1.c}
\end{minipage}

When the python program on the left is run with cpython, the program begins
executing and reaches line 5 before producing a run-time error: \code{NameError:
name 'xx' is not defined}. In contrast, the C program fails to compile, emitting
the compile-time error \code{error: use of undeclared identifier 'xx'}. If the
python code was to refer to the incorrectly named code in an infrequent code
path, the error could never be caught and reported, even with testing.

\begin{minipage}[t]{0.5\textwidth}
\lstinputlisting[language=Python]{figs/python2.py}
\end{minipage}
\begin{minipage}[t]{0.5\textwidth}
\lstinputlisting[language=C]{figs/c2.c}
\end{minipage}

These programs, much like the earlier programs, also demonstrate a common error:
passing values with incorrect types into a function. In the python code, the
error is reported at run-time on line 3 - \emph{within} the function. This
function could theoretically be located within a library which a developer is
using, meaning that the error is reported at a code location which is unrelated
to the actual programming error. In contrast, The strongly-typed C program
reports the error on line 10, which is the call site where the actual mistake
occurred.


% Problem Overview
\section{Problem Overview}

The goal of this thesis was to design, implement, and demonstrate a new teaching
programming language: Garter. Garter is intended to be a language which both
makes teaching programming concepts to new developers easy, and also provides
a smooth path for moving away from the learning environment to implementing and
solving real problems. We consider such a language to have to fit the following
requirements:

\begin{enumerate}
\item The programmer should not be required to use syntax or language features
until they understand its meaning, yet should be able to start writing code
before they have been taught about syntax.
\item The language should act as a stepping stone to a production programming
language, such that eager learners can use what they have learned in class to
write more complex programs and springboard into learning a language which is
used by professionals in both Scientific Programming and Application Programming.
\item Confusing edge cases should be eliminated, such that students and teachers
are able to safely reason about a program without having to spend time studying
the language's particulars.
\item The language should prevent incorrect programs from being written, and
provide useful explanations as to how to correct the program being run, rather
than breaking in a confusing manner at runtime.
\end{enumerate}

% Thesis Contributions
\section{Thesis Contributions}

The main contributions of this thesis are as follows:

\begin{itemize}
\item We designed a new teaching programming language, Garter, which aims to
bring together the simplicity of other teaching programming languages, like
Turing, and the more modern dynamic programming language Python.
\item We created a prototype implementation of Garter, based on the cpython
runtime, including using the cpython IDE, IDLE, and adapting it to work with
Garter code.
\item We implemented a series of example programs, in Garter, demonstrating
that it can be used to solve simple problems like those solved in classrooms,
and showing how to migrate Turing style programs to Garter.
\end{itemize}


% Thesis Outline
\section{Thesis Outline}
The remainder of this thesis is organized as follows:

\noindent\textbf{Chapter 2, Background:} An exploration of the other languages and tooling which exist in this space

\noindent\textbf{Chapter 3, Design:} A discussion of the design decisions made for the Garter language to make it fit the 4 requirements.

\noindent\textbf{Chapter 4. Prototype Implementation:} An exploration of the prototype implementation of Garter

\noindent\textbf{Chapter 5, Conclusions and Future Work:} Where Garter has succeeded and failed, as well as the path foward to a better educational programming language
