% Chapter 2
\glsresetall % reset the glossary to expand acronyms again
\chapter{Background and Related Works}\label{ch:Background}

Garter aims to simplify and restrict the dynamic programming language Python
inro a learning-friendly dialect. This is done by adding a type system, and
restricting the types of programs which can be written to make writing code simpler.

There are other projects which have aimed to add a type system to a dynamic
language, with varying levels of success. 

The most similar project to this thesis, is the mypy project \cite{mypyweb}.
Mypy is an optional static type checker for Python, which is designed to enable
Python developers with large code bases to take advantage of static typing to
make better, more maintainable software. The goal of mypy is to enable as many
python structures and patterns to be described with its type system, which means
that it possesses a relatively complex system, which includes bidirectional type
inference, generics, function types, abstract base classes, multiple inheritance
and more.

This thesis proposes a language which is much simpler and smaller than that
provided by mypy. The reason for this is that the most important and useful part
of a learning language is its error messages. As a type system gets more complex
and tries to cover all types of software which a professional programmer would
try to write, it can become unwieldy and produce unclear error messages. This
thesis intentionally maintains a small type system footprint such that it can
detect errors quickly, and provide useful errors, with suggestions for how the
student can improve their software. Mypy also attempts to provide an easy
migration path for developers with large existing python code bases: namely it
is dynamic by default, and only becomes statically typed when type annotations
are added. This is undesirable for a learning language, as we want the safer
option to be the default option, otherwise students will accidentally write
unsafe code, and not benefit from static typing and improved error detection.

Another similar project is Microsoft's Typescript
project \cite{typescriptpaper}, \cite{typescriptweb}. Much like mypy, Typescript
aims to extend the ECMAScript language \cite{ecmascript} to have optional
gradual typing. Many of the same complaints which apply to mypy also apply to
Typescript: namely that it's goal is not teaching, and it ends up as a complex
system in order to work for real production projects. By making covering all of
the capabilities of dynamically typed languages a non-goal for this thesis, we
hope to enable better error messages, and a simpler introduction to programming
for students learning with it.

The Turing programming language is a teaching programming language which
heavily inspired Garter's type system and decisions \cite{turingpaper}. It has
a simple nominal type system, and aims to provide good error messages. Turing is
used throughout Ontario as a teaching language for new students in high school
due to its simplicity. Unfortunately, Turing fails in a few areas. Firstly, its
syntax no longer feels like the syntax of many modern programming languages.
It's more low level than modern programming languages, not taking advantage of
modern programming language features which are popular like Garbage Collection,
and can help releive new developers from thinking of details, and it doesn't
follow modern programming conventions of using growable arrays, maps, and
the use of objects in order to encapsulate state.

Garter aims to take inspiration from Turing's simplicity, while extending on it
with a modern programming lens in order to design a language which helps new
programmers write good code, while also making them familiar with Python.
